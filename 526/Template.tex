\documentclass[11pt, ngerman, fleqn, DIV=15, headinclude, BCOR=2cm]{scrreprt}

\usepackage{../../header}

\usepackage{placeins}
%\usepackage[maxfloats=50]{morefloats}

\usepackage{csquotes}

\usepackage{tikz}
\usetikzlibrary{chains}
\usetikzlibrary{shapes.geometric}

\tikzset{device/.style={
                rectangle,
                minimum size=6mm,
                draw=black
            },
            monitor/.style={
                rectangle,
                rounded corners=2mm,
                minimum size=6mm,
                draw=black
            },
        }

\usepackage{pgfplots}
\pgfplotsset{
    compat=1.9,
    width=0.8\linewidth,
    xticklabel style={/pgf/number format/use comma},
    yticklabel style={/pgf/number format/use comma},
}
\usepgfplotslibrary{polar}

\usepgfplotslibrary{external}
\tikzexternalize[mode=list and make]
\tikzsetexternalprefix{Abbildung-}

\DeclareSIUnit{\skt}{SKT}

\usepackage{booktabs}

\hypersetup{
    pdftitle=
}

\newcommand{\plotwidth}{0.8\linewidth}

\subject{Praktikumsprotokoll}
\title{Compton-Effekt}
\subtitle{Versuch P526 -- Universität Bonn}
\author{
	Frederike Schrödel \\
	\small{\href{mailto:fschroedel@gmx.de}{fschroedel@gmx.de}}
	\and
	Simon Schlepphorst \\
	\small{\href{mailto:s2@uni-bonn.de}{s2@uni-bonn.de}}
}

\date{2015-11-10}

\publishers{Tutor: Peter Klassen
}

\begin{document}

\maketitle

\begin{abstract}
In diesem Versuch beschäftigen wir uns mit der Wechselwirkung von
$\gamma$-Strahlung in Materie, wobei wir uns besonders mit der
Comptonstreuung beschäftigen.
Hierbei interessieren uns vor allen der totale Wirkungsqureschnitt und die
Winkelabhängigkeit von Intensität und der Energie der $\gamma$-Strahlung.
\end{abstract}

\tableofcontents

\chapter{Theorie}
%\section{Wechselwirkung von $\gamma$-Strahlung in Materie}
%Energiereiche Photonen, wie die zu betrachtenden $\gamma$-Teilchen können in
%Materie Wechselwirken. Hierbei hängt der dominierende Effekt von der
%Energie der Teilchen und der Ordnungszahl ab.
%TODO Skizze Wirkungsquerschnitt-energie 
%Im unteren Energiebereich, ab dem Punkt wo die Energie höher ist als die
%Bindungsernergie findet hauptsächlich der Photoeffekt statt. Im mittleren
%Bereich dominiert der Compten-Effekt und ab einer Energie von
%\SI{1022}{\kilo\electronvolt} kann auch vermehrt die Elektron-Positron-
%Paarbildung auf treten.
%%%%%%%%%%%%%%%%%%%%%%%%%%%%%%%%%%%%%%%%%%%%%%%%%%%%%%%%%%%%%%%%%%%%%%%%%%%%%%%%%

\section{Wechselwirkung von $\gamma$-Strahlung mit Materie}
\label{sec:WW-strahlung-Materie}
Die $\gamma$-Strahlung entsteht, wenn nach einem vorhergegangen Zerfall der
Tochterkern nicht in seinen Grundzustand, sondern in einen angeregten Zustand
zerfällt.
Dieser angeregte Zustand zerfällt nach einiger Zeit durch Abstrahlung des
$\gamma$-Quants in den Grundzustand. Diese Photonen sind sehr Energiereich,
weshalb sie genug Energie haben, um auf verschiedene Weisen mit der Materie zu
Wechselwirken.Hierbei hängt der dominierende Effekt von der
Energie der Teilchen und der Ordnungszahl ab.
%TODO Skizze Wirkungsquerschnitt-energie 
Im unteren Energiebereich, ab dem Punkt wo die Energie höher ist als die
Bindungsernergie findet hauptsächlich der Photoeffekt statt. Im mittleren
Bereich dominiert der Compten-Effekt und ab einer Energie von
\SI{1022}{\kilo\electronvolt} kann auch vermehrt die Elektron-Positron-

\subsection{Paarerzeugung}
Ein weiterer Effekt, der bei Energien $E_\gamma \ge 2m_\text ec^2$ auftreten
kann, ist die Elektron-Positron-Paarbildung.
Dieser Effekt ist allerdings erst ab weit oberhalb dieser Schwelle relevant.
Für den Wirkungsquerschnitt gilt $\sigma \propto Z^2$.

\subsection{Photoeffekt}
Der Photoeffekt beschreibt den Vorgang, bei welchem ein Photon ein Elektron aus
dem Atom auslöst.
Hierbei gibt das Photon seine komplette Energie ab.
Die kinetische Energie des ausgelösten Elektrons hängt nur über die
Bindungsenergie ($E_\text{Bind.}$) mit dem der Energie des Photons ($h\nu$),
also dessen Frequenz ($\nu$) zusammen.
Dieser Prozess kann nur in der Nähe eines Atomkerns stattfinden, weil dieser
den Rückstoß aufnimmt.
So bleibt bei diesen Vorgang Energie und Impuls erhalten. 
\[ 
    E = h\nu - E_\text{Bind.}
\]
Wenn man sich den Wirkungsquerschnitt $\sigma$ für diesen Prozess anschaut,
dann stellt man eine große Abhängigkeit von der Ordnungszahl $Z$ fest.
\[
    \sigma \propto Z^5 E_\gamma^{\frac 72}
\]
Deshalb tritt dieser Effekt besonders häufig bei Elementen mit hoher Ordnungszahl
auf.
 
\subsection{Comptonstreuung}
Wenn Photonen an einem Elektron streuen, dann spricht man vom Comptoneffekt. 
Hierbei gibt das Photon einen Teil seiner an das Elektron ab und ändert dabei
seine Frequenz.%TODO bild Comptonstreuung
Die abgegebene Energie hängt dabei vom Streuwinkel ab.
Für die Restenergie des Photons gilt:
\[
    E_\nu'(\phi) = \frac{E_\nu}{1+\frac{E_\nu}{m_\text ec^2}(1+\cos(\phi))}
\]
Die größte Energieänderung erhält man also bei einem Winkel von
$\phi=\SI{180}{\degree}$.
Diese Energieänderung berechnet sich durch:
\[
    \Delta\lambda = \frac{h}{m_\text ec}(1-\cos(\phi))
\]
Der Wirkungsquerschnitt entspricht $\sigma_\text C \propto \frac{Z}{E_\gamma}$.
%TODO nur für coptoneffekt
%TODO bild Spektrum im szintilator

\subsection{Keine Wechselwirkung}
Ein nicht unerheblicher Teil der $\gamma$-Strahlung kann auch Wechselwirkung
Materie durchdringen.

%TODO compton
\section{Wirkungsquerschnitte}
Hier muss zwischen dem differenziellen- und dem absoluten Wirkungsquerschnitt
unterschieden werden.
Die Herleitung für den Wirkungsquerschnitt des Comptoneffekt gelang untern
Nutzung der Dirac-Gleichung. Verantwortlich hierfür waren Nishina und
Klein, nach welchen auch die Darstellung, der Klein-Nishina-Plot, benannt ist.
\[
    \dod \sigma\Omega = \frac{r^2_\text{e}}{2}
    \frac{1}{(1+\gamma(1-\cos(\theta)))^2}\del{1+\cos(\theta)^2+\frac{\gamma^2(1-\cos(\theta)^2)}{1+\gamma(1-\cos(\theta))}}
\]
\label{eq:dif-querschnitt}
\ref{eq:dif-querschnitt} heißt Klein-Nishina Formel und beschreibt den
differenziellen Wirkungsquerschnitt mit dem Stoßwinkel $\theta$.
Um den totalen Wirkungsquerschnitt zu erhalten integriert man über dem
Raumwinkel:
\[
    \sigma = 2\pi r^2_e
    \del{\frac{1+\gamma}{\gamma^2}\del{\frac{2(1+\gamma)}{1+2\gamma}-\frac
        1\gamma\ln(1+2\gamma)}+\frac
    1{2\gamma}\ln(1+2\gamma)-\frac{1+3\gamma}{(1+2\gamma)^2}}
\]
%Wenn man hieraus das Energiespektrum ermitteln möchte, so kann man in der
%Klein-Nishina-Formel 
%\[
%    \dod{\sigma}{T} = 
%    \frac{\pi r^2_\text e}{m_\text ec^2\gamma^2} \del{
%        2+\frac{s^2}{\gamma^2(1-s)^2}+\frac{s}{1-s}\del{s-\frac 2\gamma}}
%\]

Darstellen lässt sich die Winkelverteilung der Photonen durch den Wirkungsquerschnitt in einem
Klein-Nishina Plot. 
%TODO klein nishina plot

\section{Szintillationsdetektor}
Mit einem Szintillationsdetektor lassen sich Intensität und Energie von
ionisierender Strahlung messen.
Er besteht aus einem Szintillator und einem Photomultiplier.
Der Szintillator ist ein dotierter Einkristall, der von energiereicher
Strahlung zum Fluoreszieren angeregt werden kann und hinreichend
durchlässig für das von ihm emittierte Licht ist.

Innerhalb des vor Lichteinfall geschützten Szintillators entstehen durch
ionisierende Strahlung Lichtblitze. 
Diese entstehen dadurch, dass die Strahlung Atome durch die Effekt
~\ref{sec:WW-strahlung-Materie} angeregt, welche entweder direkt
wieder rekombinieren und Photonen aussenden oder über Stöße vorher einen Teil
ihrer Energie abgeben.
Somit ist die Intesität dieser Blitze abhängig von der Energie der einfallenden Strahlung.
%TODO nur für coptoneffekt? Bild Spektrum
In dem entstehenden Spektrum, bei dem die Intensität gegen Energie
aufgetragen. Man erhält ein Compton-Kontinuum, welches die Comptonstreuung bis
zu einem Winkel von \SI{180}{\degree} enthält. Da bei diesem Wert der maximale
Energieübertrag stattfindet, hat erhält man eine scharfe Kante, die Compton
Kante. Wird nicht nur ein Teil der Energie abgegeben, wie es bei Photoeffekt
der Fall ist, entsteht der energetisch höher gelegene Ausschlag. Man nennt
ihn Photo-Peak.



\subsection{Photomultiplier}
Aufgrund des Photoeffektes, werden an der Photokathode des Photomultipliers
Elektronen ausgelöst.

Durch den Aufbau weitern des Photomultiplieres kommt es zu einem Lawineneffekt,
welcher die Elektronen vervielfacht und somit eine Detektion erleichtert.
Die Amplitude des so entstanden Strompulses ist somit auch proportional zu der
Energie der Eingangsstrahlung.
%TODO für Driftkammer
%Wenn man eine gute Energieauflösung möchte, sollte man allerdings eher einem
%Halbleiterdetektor nutzen.

\subsection{Energie und Zeitauflösung}
Um mit dieser Art Detektor möglichst gute Auflösungen zu erhalten, greift man
das Signal je nachdem ob man die Energie- oder die Zeitauflösung optimieren will
an eine Dynode beziehungsweise an der Anode des Photomultipliers ab.
Der Vorteil der Dynode für die Energieauflösung liegt da drin, dass noch keine
Sättigung herrscht. Hierdurch ist die Proportionalität zwischen Signalhöhe und
Energie nicht beeinträchtigt. Da das Signal allerdings nur sehr langsam
ansteigt, man spricht von den Slow-Signal, ist die Zeitauflösung beeinträchtigt.
Deswegen benutzt man hier das sogenannte Fast-Signal. Es wird an der Anode
abgegriffen, da hier Sättigung vorliegt, wodurch das Signal schnell ansteigt.
Die Amplitude ist allerdings nicht mehr proportional zur Energie.
Um die Zeitauflösung ermitteln zu können nutzt man eine
Slow-Fast-Koinzidenzschaltung. Man misst bekannte Linien und bestimmt aus den
angepassten Funktionen die Halbwertsbreite.

%TODO für Nukleare elektronik
%Für die Zeitkalibrirung nutzt man
%am besten ein Isotope, welches einen $\beta^+$-Zerfall besitzt. Dadurch kann
%man gewährleisten, dass gleichzeitig zwei Photonen mit bekannter Wellenlänge
%von \SI{511}{\ilko\electronenvot} durch die Rekombination von Elektron und
%Positron entstehen.
\chapter{Durchführung}

\section{Einstellen des Verstärkers}
\begin{figure}[h]
    \centering
    \includegraphics[width=\plotwidth]{plot_fit_peak_ohne}
    \caption{%
	    $^{137}\text{Cs}$-Spektrum ohne Aluminium Absorber
   }
    \label{fig:plot_fit_peak_ohne}
\end{figure}

%TODO

\chapter{Auswertung}

\section{Totaler Wirkungsquerschnitt}

Es werden die Messdaten aus \fehlt%TODO\ref{}
benutzt.
An die Photopeaks der Spektren passen wir eine Gaussfunktion an. Das Resultat
ist vergrößert in Abbildung~\ref{fig:amplituden} zu sehen. Die gesamten
Spektren kann man in Abbildung~\ref{fig:plot_fit_peak_ohne} und im
Anhang~\ref{anhang-wirkungsquerschnitt} betrachten.

\begin{equation}
	g\del x = \frac{a}{\varsigma \sqrt{2 \pi}} \exp\del{\frac{\del
		{x - x_0}^2}{2 \varsigma^2}}
		\quad \quad \quad \text{mit FWHM} \approx 2.355 \varsigma
\end{equation}

Die Flächen unter den Gaußkurven sind auf die Kurve ohne Absorber normiert als
Intensitäten in Tabelle~\ref{tab:amplituden} aufgeführt. An diese wurde in
Abbildung~\ref{fig:cross-section} eine Exponentialfunktion der folgenden Form
angepasst:

\begin{equation}
	N\del x = N_0 \exp\del{-\sigma \rho_T x} \quad \quad \quad \text{mit }
	\rho_T = \frac{N_A \rho}{M}
\end{equation}

Mit dem Ergenis aus der Anpassung und den Literaturwerten für Aluminium ($\rho
= \SI{<< alu_dichte >>}{\gram\per\cubic\centi\metre}$, $M = \SI{<< alu_amu >>}
{\atomicmassunit}$) ergibt sich für den totalen Wirkungsquerschnitt $\sigma$:

\begin{align*}
	&\sigma \rho_T = \SI{<< alu_a >>}{\per\milli\metre}
	&&\rho_T = \SI{<< alu_n >>}{\per\cubic\milli\metre}
	&&\implies \sigma = \SI{<< alu_sigma >>}{\barn}
\end{align*}

\fehlt %TODO Diskussion, Detektoreffizienz nicht beachtet

\begin{figure}
    \centering
    \includegraphics[width=\plotwidth]{total-cross-section-data}
    \caption{%
	    Photopeaks im Spektrum der eingebauten $^{137}\text{Cs}$-Quelle ohne
	    und mit \SIlist{1;5;10;20}{\milli\meter} Aluminium Absorber
    }
    \label{fig:amplituden}
\end{figure}

\begin{table}
    \centering
    \begin{tabular}{SSSS}
        {Dicke / \si{\milli\meter}} &
        {Scheitelpunkt / Kanal} &
        {FWHM / Kanal} &
	{rel. Intensität} \\
        \midrule
        %< for row in total_cross_section_table: ->%
        << ' & '.join(row) >> \\
        %< endfor ->%
    \end{tabular}
    \caption{%
        Anpassungsparameter für die verschiedenen Dicken der
        Absorbermaterialien.
    }
    \label{tab:amplituden}
\end{table}

\begin{figure}
    \centering
    \includegraphics[width=\plotwidth]{total-cross-section-fit}
    \caption{%
	    Transmittierte relative Intensität abhängig von der Absoberdicke
	    mit angepasster Exponentialfunktion
    }
    \label{fig:cross-section}
\end{figure}

\clearpage

\section{Energiekalibrierung}

Mit den Messdaten aus \fehlt%TODO\ref{ }
und der Messung ohne Absober aus \fehlt%TODO\ref{ }
wird nun die Energiekalibrierung durchgeführt. Dazu werden an die Spitzen der
Spektren Gaussfunktionen angepasst. Das Ergebnis ist in
Abbildung~\ref{fig:plot_fit_peak_ohne} und den Abbildungen im
Anhang~\ref{anhang-energiekalibrierung}
zu sehen. Außerdem sind die Parameter der Gaußfunktionen nocheinmal in
Tabelle~\ref{tab:energiekalibrierung}
aufgeführt. Mit diesen Werten lässt sich nun eine Energie-Kanal Beziehung
erstellen:
\begin{align}
	E\del x = m x + n
	&&\text{mit } m = \SI{<< energy_slope >>}{\kilo\electronvolt}\text{, }
	\quad n = \SI{<< energy_offset >>}{\kilo\electronvolt}
\end{align}

\begin{figure}
	\centering
	\begin{tabular}{SSS}
		{Scheitelpunkte / Kanal} &
		{FWHM / Kanal} &
		{Energie / \si{\kilo\electronvolt}}\\
		\midrule
		%< for row in energy_calibration_table: ->%
		<< ' & '.join(row) >> \\
		%< endfor ->%
	\end{tabular}
	\caption{%
		Anpassungeparameter für die Energiekalibrierung
	}
	\label{tab:energiekalibrierung}
\end{figure}

\begin{figure}
    \centering
    \includegraphics[width=\plotwidth]{plot_energy-calibrate-fit}
    \caption{%
	    Energiekalibrierung
    }
    \label{fig:plot_energy-calibrate-fit}
\end{figure}


%TODO

\chapter{Ergebnis}

%TODO


%%%%%%%%%%%%%%%%%%%%%%%%%%%%%%%%%%%%%%%%%%%%%%%%%%%%%%%%%%%%%%%%%%%%%%%%%%%%%%%
%                                   Anhang                                    %
%%%%%%%%%%%%%%%%%%%%%%%%%%%%%%%%%%%%%%%%%%%%%%%%%%%%%%%%%%%%%%%%%%%%%%%%%%%%%%%

\begin{appendix}

\chapter{Anhang}

\section{Abbildungen von Spektren mit Untergrund}\label{anhang-rohspektren-energiekalibrierung}
\begin{figure}[h]
    \centering
    \includegraphics[width=0.65\textwidth]{plot_data_raw_Ba}
    \caption{%
	    $^{133}\text{Ba}$-Spektrum mit Untergrund
    }
    \label{fig:plot_data_raw_Ba}
\end{figure}

\begin{figure}[h]
    \centering
    \includegraphics[width=0.65\textwidth]{plot_data_raw_Eu}
    \caption{%
	    $^{152}\text{Eu}$-Spektrum mit Untergrund
   }
    \label{fig:plot_data_raw_Eu}
\end{figure}

\clearpage

\section{Abbildungen zur Bestimmung des totalen Wirkungsquerschnitts} \label{anhang-wirkungsquerschnitt}
\begin{figure}[h]
    \centering
    \includegraphics[width=0.65\textwidth]{plot_fit_peak_01mm}
    \caption{%
	    $^{137}\text{Cs}$-Spektrum mit \SI{1}{\milli\metre} Aluminium
	    Absorber
    }
    \label{fig:plot_fit_peak_01mm}
\end{figure}

\begin{figure}[h]
    \centering
    \includegraphics[width=0.65\textwidth]{plot_fit_peak_05mm}
    \caption{%
	    $^{137}\text{Cs}$-Spektrum mit \SI{5}{\milli\metre} Aluminium
	    Absorber
   }
    \label{fig:plot_fit_peak_05mm}
\end{figure}

\begin{figure}[h]
    \centering
    \includegraphics[width=0.65\textwidth]{plot_fit_peak_10mm}
    \caption{%
	    $^{137}\text{Cs}$-Spektrum mit \SI{10}{\milli\metre} Aluminium
	    Absorber
   }
    \label{fig:plot_fit_peak_10mm}
\end{figure}

\begin{figure}[h]
    \centering
    \includegraphics[width=0.65\textwidth]{plot_fit_peak_20mm}
    \caption{%
	    $^{137}\text{Cs}$-Spektrum mit \SI{20}{\milli\metre} Aluminium
	    Absorber
   }
    \label{fig:plot_fit_peak_20mm}
\end{figure}

\clearpage

\section{Abbildungen zur Energiekalibrierung} \label{anhang-energiekalibrierung}
\begin{figure}[h]
    \centering
    \includegraphics[width=0.65\textwidth]{plot_peaks_Ba}
    \caption{%
	    Untergrundbereinigtes $^{133}\text{Ba}$-Spektrum mit
	    Gaussanpassungen
   }
    \label{fig:plot_peaks_Ba}
\end{figure}

\begin{figure}[h]
    \centering
    \includegraphics[width=0.65\textwidth]{plot_peaks_Eu}
    \caption{%
	    Untergrundbereinigtes $^{152}\text{Eu}$-Spektrum mit
	    Gaussanpassungen
    }
    \label{fig:plot_peaks_Eu}
\end{figure}




%TODO

\end{appendix}

\end{document}

% vim: spell spelllang=de tw=79
