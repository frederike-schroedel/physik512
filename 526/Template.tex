\documentclass[11pt, ngerman, fleqn, DIV=15, headinclude, BCOR=2cm]{scrreprt}

\usepackage{../../header}

\usepackage{placeins}
%\usepackage[maxfloats=50]{morefloats}

\usepackage{csquotes}

\usepackage{tikz}
\usetikzlibrary{chains}
\usetikzlibrary{shapes.geometric}

\tikzset{device/.style={
                rectangle,
                minimum size=6mm,
                draw=black
            },
            monitor/.style={
                rectangle,
                rounded corners=2mm,
                minimum size=6mm,
                draw=black
            },
        }

\usepackage{pgfplots}
\pgfplotsset{
    compat=1.9,
    width=0.8\linewidth,
    xticklabel style={/pgf/number format/use comma},
    yticklabel style={/pgf/number format/use comma},
}
\usepgfplotslibrary{polar}

\usepgfplotslibrary{external}
\tikzexternalize[mode=list and make]
\tikzsetexternalprefix{Abbildung-}

\DeclareSIUnit{\skt}{SKT}

\usepackage{booktabs}

\hypersetup{
    pdftitle=
}

\newcommand{\plotwidth}{0.8\linewidth}

\subject{Praktikumsprotokoll}
\title{Compton-Effekt}
\subtitle{Versuch P526 -- Universität Bonn}
\author{
	Frederike Schrödel \\
	\small{\href{mailto:fschroedel@gmx.de}{fschroedel@gmx.de}}
	\and
	Simon Schlepphorst \\
	\small{\href{mailto:s2@uni-bonn.de}{s2@uni-bonn.de}}
}

\date{2015-11-10}

\publishers{Tutor: Peter Klassen
}

\begin{document}

\maketitle

\begin{abstract}
%TODO
\end{abstract}


\tableofcontents

\chapter{Theorie}

%TODO

\chapter{Durchführung}

\section{Einstellen des Verstärkers}
\begin{figure}[h]
    \centering
    \includegraphics[width=\plotwidth]{plot_fit_peak_ohne}
    \caption{%
	    $^{137}\text{Cs}$-Spektrum ohne Aluminium Absorber
   }
    \label{fig:plot_fit_peak_ohne}
\end{figure}

%TODO

\chapter{Auswertung}

\begin{figure}
    \centering
    \includegraphics[width=\plotwidth]{total-cross-section-data}
    \caption{%
	    Photopeaks im Spektrum der eingebauten $^{137}\text{Cs}$-Quelle ohne
	    und mit \SIlist{1;5;10;20}{\milli\meter} Aluminium Absorber
    }
    \label{fig:amplituden}
\end{figure}

\begin{table}
    \centering
    \begin{tabular}{SSSS}
        {Dicke / \si{\milli\meter}} &
        {Scheitelpunkt / Kanal} &
        {FWHM / Kanal} &
	{rel. Intensität} \\
        \midrule
        %< for row in total_cross_section_table: ->%
        << ' & '.join(row) >> \\
        %< endfor ->%
    \end{tabular}
    \caption{%
        Anpassungsparameter für die verschiedenen Dicken der
        Absorbermaterialien.
    }
    \label{tab:amplituden}
\end{table}



\begin{figure}
    \centering
    \includegraphics[width=\plotwidth]{total-cross-section-fit}
    \caption{%
    }
    \label{fig:cross-section}
\end{figure}

%TODO

\chapter{Ergebnis}

%TODO


%%%%%%%%%%%%%%%%%%%%%%%%%%%%%%%%%%%%%%%%%%%%%%%%%%%%%%%%%%%%%%%%%%%%%%%%%%%%%%%
%                                   Anhang                                    %
%%%%%%%%%%%%%%%%%%%%%%%%%%%%%%%%%%%%%%%%%%%%%%%%%%%%%%%%%%%%%%%%%%%%%%%%%%%%%%%

\begin{appendix}

\chapter{Anhang}

\section{Abbildungen zur Bestimmung des totalen Wirkungsquerschnitts} \label{anhang-wirkungsquerschnitt}
\begin{figure}[h]
    \centering
    \includegraphics[width=0.6\textwidth]{plot_fit_peak_01mm}
    \caption{%
	    $^{137}\text{Cs}$-Spektrum mit \SI{1}{\milli\metre} Aluminium
	    Absorber
    }
    \label{fig:plot_fit_peak_01mm}
\end{figure}

\begin{figure}[h]
    \centering
    \includegraphics[width=0.6\textwidth]{plot_fit_peak_05mm}
    \caption{%
	    $^{137}\text{Cs}$-Spektrum mit \SI{5}{\milli\metre} Aluminium
	    Absorber
   }
    \label{fig:plot_fit_peak_05mm}
\end{figure}

\begin{figure}[h]
    \centering
    \includegraphics[width=0.6\textwidth]{plot_fit_peak_10mm}
    \caption{%
	    $^{137}\text{Cs}$-Spektrum mit \SI{10}{\milli\metre} Aluminium
	    Absorber
   }
    \label{fig:plot_fit_peak_10mm}
\end{figure}

\begin{figure}[h]
    \centering
    \includegraphics[width=0.6\textwidth]{plot_fit_peak_20mm}
    \caption{%
	    $^{137}\text{Cs}$-Spektrum mit \SI{20}{\milli\metre} Aluminium
	    Absorber
   }
    \label{fig:plot_fit_peak_20mm}
\end{figure}

%TODO

\end{appendix}

\end{document}

% vim: spell spelllang=de tw=79
