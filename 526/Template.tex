\documentclass[11pt, ngerman, fleqn, DIV=15, headinclude, BCOR=2cm]{scrreprt}

\usepackage{../../header}

\usepackage{placeins}
%\usepackage[maxfloats=50]{morefloats}

\usepackage{csquotes}

\usepackage{tikz}
\usetikzlibrary{chains}
\usetikzlibrary{shapes.geometric}

\tikzset{device/.style={
                rectangle,
                minimum size=6mm,
                draw=black
            },
            monitor/.style={
                rectangle,
                rounded corners=2mm,
                minimum size=6mm,
                draw=black
            },
        }

\usepackage{pgfplots}
\pgfplotsset{
    compat=1.9,
    width=0.8\linewidth,
    xticklabel style={/pgf/number format/use comma},
    yticklabel style={/pgf/number format/use comma},
}
\usepgfplotslibrary{polar}

\usepgfplotslibrary{external}
\tikzexternalize[mode=list and make]
\tikzsetexternalprefix{Abbildung-}

\DeclareSIUnit{\skt}{SKT}

\usepackage{booktabs}

\hypersetup{
    pdftitle=
}

\newcommand{\plotwidth}{0.8\linewidth}

\subject{Praktikumsprotokoll}
\title{Compton-Effekt}
\subtitle{Versuch P526 -- Universität Bonn}
\author{
	Frederike Schrödel \\
	\small{\href{mailto:fschroedel@gmx.de}{fschroedel@gmx.de}}
	\and
	Simon Schlepphorst \\
	\small{\href{mailto:s2@uni-bonn.de}{s2@uni-bonn.de}}
}

\date{2015-11-10}

\publishers{Tutor: Peter Klassen
}

\begin{document}

\maketitle

\begin{abstract}
In diesem Versuch beschäftigen wir uns mit der Wechselwirkung von
$\gamma$-Strahlung in Materie, wobei wir uns besonders mit der
Comptonstreuung beschäftigen.
Hierbei interessieren uns vor allen der totale Wirkungsqureschnitt und die
Winkelabhängigkeit von Intensität und der Energie der $\gamma$-Strahlung.
\end{abstract}

\tableofcontents

\chapter{Theorie}
\section{Wechselwirkung von $\gamma$-Strahlung in Materie}
Energiereiche Photonen, wie die zu betrachtenden $\gamma$-Teilchen können in
Materie Wechselwirken. Hierbei hängt der dominierende Effekt von der
Energie der Teilchen und der Ordnungszahl ab.
%TODO Skizze Wirkungsquerschnitt-energie 
Im unteren Energiebereich, ab dem Punkt wo die Energie höher ist als die
Bindungsernergie findet hauptsächlich der Photoeffekt statt. Im mittleren
Bereich dominiert der Compten-Effekt und ab einer Energie von
\SI{1022}{\kilo\electronvolt} kann auch vermehrt die Elektron-Positron-
Paarbildung auf treten.



\chapter{Durchführung}

%TODO

\chapter{Auswertung}

%TODO

\chapter{Ergebnis}

%TODO


%%%%%%%%%%%%%%%%%%%%%%%%%%%%%%%%%%%%%%%%%%%%%%%%%%%%%%%%%%%%%%%%%%%%%%%%%%%%%%%
%                                   Anhang                                    %
%%%%%%%%%%%%%%%%%%%%%%%%%%%%%%%%%%%%%%%%%%%%%%%%%%%%%%%%%%%%%%%%%%%%%%%%%%%%%%%

\begin{appendix}

%TODO

\end{appendix}

\end{document}

% vim: spell spelllang=de tw=79
