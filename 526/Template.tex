\documentclass[11pt, ngerman, fleqn, DIV=15, headinclude, BCOR=2cm]{scrreprt}

\usepackage{../../header}

\usepackage{placeins}
%\usepackage[maxfloats=50]{morefloats}

\usepackage{csquotes}

\usepackage{tikz}
\usetikzlibrary{chains}
\usetikzlibrary{shapes.geometric}

\tikzset{device/.style={
                rectangle,
                minimum size=6mm,
                draw=black
            },
            monitor/.style={
                rectangle,
                rounded corners=2mm,
                minimum size=6mm,
                draw=black
            },
        }

\usepackage{pgfplots}
\pgfplotsset{
    compat=1.9,
    width=0.8\linewidth,
    xticklabel style={/pgf/number format/use comma},
    yticklabel style={/pgf/number format/use comma},
}
\usepgfplotslibrary{polar}

\usepgfplotslibrary{external}
\tikzexternalize[mode=list and make]
\tikzsetexternalprefix{Abbildung-}

\DeclareSIUnit{\skt}{SKT}

\usepackage{booktabs}

\hypersetup{
    pdftitle=
}

\newcommand{\plotwidth}{0.8\linewidth}

\subject{Praktikumsprotokoll}
\title{Compton-Effekt}
\subtitle{Versuch P526 -- Universität Bonn}
\author{
	Frederike Schrödel \\
	\small{\href{mailto:fschroedel@gmx.de}{fschroedel@gmx.de}}
	\and
	Simon Schlepphorst \\
	\small{\href{mailto:s2@uni-bonn.de}{s2@uni-bonn.de}}
}

\date{2015-11-10}

\publishers{Tutor: Peter Klassen
}

\begin{document}

\maketitle

\begin{abstract}
%TODO
\end{abstract}


\tableofcontents

\chapter{Theorie}

%TODO

\chapter{Durchführung}

%TODO

\chapter{Auswertung}

\begin{figure}
    \centering
    \tikzsetnextfilename{amplituden}
    \begin{tikzpicture}
        \begin{axis}[
                width=\linewidth,
                height=0.6\linewidth,
                xlabel=Kanal,
                ylabel=Ereignisse,
                grid=major,
            ]
            \addplot[gray] table {plot_data_00mm.txt};
            %\addplot[black] table {plot-decay-used-00mm.txt};

            \addplot[gray] table {plot_data_01mm.txt};
            %\addplot[black] table {plot-decay-used-01mm.txt};

            \addplot[gray] table {plot_data_05mm.txt};
            %\addplot[black] table {plot-decay-used-05mm.txt};

            \addplot[gray] table {plot_data_10mm.txt};
            %\addplot[black] table {plot-decay-used-10mm.txt};

            \addplot[gray] table {plot_data_20mm.txt};
            %\addplot[black] table {plot-decay-used-20mm.txt};


            \addplot[red, thick] table {plot_fit_00mm.txt};
            \addplot[red, thick] table {plot_fit_01mm.txt};
            \addplot[red, thick] table {plot_fit_05mm.txt};
            \addplot[red, thick] table {plot_fit_10mm.txt};
            \addplot[red, thick] table {plot_fit_20mm.txt};

        \end{axis}
    \end{tikzpicture}
    \caption{%
        Spektrum der eingebauten Quelle ohne und mit
        \SIlist{1;5;10;20;50}{\milli\meter} Aluminium Absorber.
    }
    \label{fig:amplituden}
\end{figure}

\begin{figure}
    \centering
    \tikzsetnextfilename{decay}
    \begin{tikzpicture}
        \begin{axis}[
                width=\linewidth,
                height=0.6\linewidth,
                xlabel=Kanal,
                ylabel=Integral,
                grid=major,
            ]
            \addplot[
                black,
                mark=|,
                only marks,
                error bars/.cd,
                y dir=both, y explicit,
            ]
            table[y error index=2] {plot-decay-data.txt};
            \addplot[red] table {plot-decay-fit.txt};
        \end{axis}
    \end{tikzpicture}
    \caption{%
    }
    \label{fig:decay}
\end{figure}

%TODO

\chapter{Ergebnis}

%TODO


%%%%%%%%%%%%%%%%%%%%%%%%%%%%%%%%%%%%%%%%%%%%%%%%%%%%%%%%%%%%%%%%%%%%%%%%%%%%%%%
%                                   Anhang                                    %
%%%%%%%%%%%%%%%%%%%%%%%%%%%%%%%%%%%%%%%%%%%%%%%%%%%%%%%%%%%%%%%%%%%%%%%%%%%%%%%

\begin{appendix}

%TODO

\end{appendix}

\end{document}

% vim: spell spelllang=de tw=79
