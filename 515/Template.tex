\documentclass[11pt, ngerman, fleqn, DIV=15, headinclude, BCOR=2cm]{scrreprt}

\usepackage{../../header}

\usepackage{placeins}
%\usepackage[maxfloats=50]{morefloats}

\usepackage{csquotes}

\usepackage{tikz}
\usetikzlibrary{chains}
\usetikzlibrary{shapes.geometric}

\tikzset{device/.style={
                rectangle,
                minimum size=6mm,
                draw=black
            },
            monitor/.style={
                rectangle,
                rounded corners=2mm,
                minimum size=6mm,
                draw=black
            },
        }

\usepackage{pgfplots}
\pgfplotsset{
    compat=1.9,
    width=0.8\linewidth,
    xticklabel style={/pgf/number format/use comma},
    yticklabel style={/pgf/number format/use comma},
}
\usepgfplotslibrary{polar}

\usepgfplotslibrary{external}
\tikzexternalize[mode=list and make]
\tikzsetexternalprefix{Abbildung-}

\DeclareSIUnit{\skt}{SKT}

\usepackage{booktabs}

\hypersetup{
    pdftitle=
}

\subject{Praktikumsprotokoll}
\title{Driftkammer}
\subtitle{Versuch P515 -- Universität Bonn}
\author{
	Frederike Schrödel \\
	\small{\href{mailto:fschroedel@gmx.de}{fschroedel@gmx.de}}
	\and
	Simon Schlepphorst \\
	\small{\href{mailto:s2@uni-bonn.de}{s2@uni-bonn.de}}
}

\date{2015-11-02}

\publishers{Tutor: Dr. Jürgen Hannappel
}

\begin{document}

\maketitle

\begin{abstract}
Ziel des Driftkammerversuches ist es, die Funktionsweise und Datengewinnung aus einer Driftkammer zu verstehen.
Hierzu nehmen wir dem Prototyp einer Driftkammer in Betrieb, 
bestimmen dessen Ansprechverhalten bei verschiedenen Hochspannungen mit oder ohne Bestrahlung durch ein $\beta$-Strahler  
und ermitteln seine Orts-Driftzeitbeziehung.
\end{abstract}

\tableofcontents

\chapter{Theorie}

\section{Einführung}
In der modernen Physik spielen Teilchenbeschleuniger eine wichtige Rolle und mit ihnen auch verschiedene Teilchendetektoren.
Wir beschäftigen uns im folgenden mit einer Driftkammer als Beispiel für ein gasbasierter Teilchendetektor.
Aufgabe der Driftkammer ist es den Ort eines bewegten Teilchen zu bestimmen.
In diesen Versuch werden die charaktristischen Eigenschaften des Prototypen einer Driftkammer des BGO-OD-Experimentes untersucht.
Hierbei wird zum einen das Verhalten der Driftkammer mit und ohne
$^90\text{Sr}$-Präperat unter verschiedenen Parametern verglichen.
Außerdem wird eine Langzeitmessung ohne Präperat durchgeführt, bei der die Winkelverteilung der kosmischen Strahlung vermessen wird.

\section{Theorie}
\subsection{Aufbau der Driftkammer}
Die Driftkammer besteht aus mit Agon-Kohlenstoffdioxidgemisch (im Verhältnis $ 82\%-18\% $ ) gefüllten Kammer, welche mit Drähten durchspannt ist. 
Hierbei bilden jeweils sechs hexagonal angeordnten Potentialdrähte, an denen
eine hohe negative spannung anliegt, und der in der Mitte liegede geerdete Anodendraht eine Driftzelle.
Die Potentialdrähte erfüllen hierbei die Aufgabe ein zylindrisches Feld zu erzeugen. 
Dieses soll die, durch Ionisierung frei werdenen Elektronen in die Richtung
des Anodendraht zu beschleunigen und verhindern, dass das Elektron in einer anderen Zelle detektiert wird.
Die Spannung der Potentialdrähte kann zwischen \SIrange{0}{3000}{\volt} variiert werden.

Ordnet man die einzelnen Driftzellen wie in Abbildung \fehlt an, so erhält man eine Driftkammer.
Da das Verhältnis der Gaszusammensetzung, dessen Druck und Temperatur
entscheidend für die Driftgeschwindigkeit der Elektronen ist, wird das Gasgemisch kontinuierlich getauscht. 
Die longitudinale Diffusion soll dabei möglichst klein gehalten werden,
anderenfalls wird die Ortsrekonstruktion ungenauer.
Das hat den Grund, dass die Longitudinale Diffusion die Zeitmessung beeinträchtigt.
Um durch die Driftgeschwindigkeit nicht durch die Bewegung des Gases stark zu verfälschen läuft dieser Vorgang sehr langsam ab.
% \fehlt bluber

\subsection{Ionisation der Gasatome}
Die Driftkammer ist der Prototyp zu einer Driftkammer die benutzt wird, um
die Bewegung schwerer geladener Teilchen zu detektieren.
Um den Energieverlust dieser sehr schnellen, geladenen Teilchen zu bestimmen kann man die Bethe-Bloch-Formel nutzen.
\[
    -\dod Ex = \frac{4 \pi r^2_e m_e c^2 N_\text{A} Z z^2}{A \beta^2}\sbr{\ln\sbr{\frac{2m_e c^2 \beta^2}{(1-\beta^2)I}}-\beta^2}
\]
Wobei $N_\text A$ die Avogadro-Zahl, $\beta c$ die Geschwindigkeit und $z$ die Ladung des Teilchens. $A$ und $Z$ sind die Massen- beziehungsweise Ordnungszahl, welche näherungsweise proportional ist zu dem Ionisationspotential $I$.

Die Abgabe der Energie findet hauptsächlich durch Ionisation statt.
Dies nennt man Primärionisation.
Die Anzahl der Primärionisationen hängt hierbei von dem Gas ab.
Bei diesen Vorgang frei werdene Energie ist größer als die Energie die für weitere Ionisationen gebraucht wird, wodurch es zu Sekundärionisationen kommt.

\subsection{Gasverstärkung, Signalentstehung und Ortsmessung}
Die freien Elektronen werden, wie oben beschrieben, aufgrund der Driftzellenengeomertie Richtung Anodendraht beschleunigt. 
Nahe des Anodendrahtes ist das elektische Feld so stark, dass die beschleunigten Elektronen weitere Gasatome ionisieren können.
So kommt es zu einen Lawineneffekt.
Das Signal wird so um einen Faktor von \numrange{e5}{e6} verstärkt.

Die Driftzeit ermittelt man, indem man die Zeitdifferenz zwischen den Durchgang durch den sehr schnell reagierenden Szintilators und dem in der Driftkammer entstandenen Signal misst.
Mit Hilfe der Orts-Driftzeitbeziehung, auch ODB genannt, lässt sich aus der Driftzeit der Ort des Teilchendurchgangs ermitteln.

\fehlt %TODO mittlere freie Weglänge
%TODO Ortsbestimmung

\chapter{Bestimmung der Betriebsparameter}

\section{Aufnahme analoger Größen}
Zunächst soll eine Hochspannung an der Driftkammer ermittelt werden, mit 
der sowohl die
Strahlung des $ ^{90}\text{Sr}$-Präperat, als auch die kosmische Strahlung
detektiert werden kann.
Hierzu wird der Kammerstrom  in Abhängigkeit von der Hochspannung, die an den
Potentialdrähten anliegt, gemessen. 
Dies wird dadurch realisiert, dass die über einem \SI{1}{\mega\ohm} Widerstand
abfallende  Spannung aus der Driftkammer gemessen wird.
Da die Kabel und die Driftkammer eine Kapazität aufweisen, muss nach jeder Änderung der
Hochspannung gewartet werde, bis die Spannung sich auf einen Wert eingestellt
hat, bevor die Messung beginnen kann.

\begin{figure}
    \centering
    \includegraphics[width=0.8 \linewidth]{m01_01_151026.pdf}
    \caption{%
	    Messung analoger Größen mit defektem Aufbau
   }
    \label{fig:m01_01-messdaten}
\end{figure}
\begin{figure}
    \centering
    \includegraphics[width=0.8 \linewidth]{m01_01_151026_2.pdf}
    \caption{%
	    Relevanz der Messung analoger Größen mit defektem Aufbau
   }
    \label{fig:m01_01-relevanz}
\end{figure}


Bei dem Vergleich der beiden Kennlinien, Kammerstrom gegen Hochspannung,
des Versuchs mit und ohne Bestrahlung fiel auf, dass es zunächst keinen
erkennbaren unterschied der Kennlinien gab.
Nach Überprüfung des Aufbaus, stellte sich raus, dass die Kammer nicht mit dem 
gewünschten Gasgemisch gefüllt war.

Nach dem die Kammer über Nacht mit dem Gasgemisch gespült wurde, lieferte der
Versuch dirket sinnvolle Ergebnisse.

\begin{figure}
    \centering
    \includegraphics[width=0.8 \linewidth]{m01_02_151027.pdf}
    \caption{%
	    Messung analoger Größen
    }
    \label{fig:m01_02-messdaten}
\end{figure}
\begin{figure}
    \centering
    \includegraphics[width=0.8 \linewidth]{m01_02_151027_2.pdf}
    \caption{%
	    Relevanz der Messung analoger Größen
   }
    \label{fig:m01_02-relevanz}
\end{figure}

Bei niedrigen Spannungen ist lassen sich die Ereignisse nicht vom Hintergrund
trennen. Dies lässt sich auf den ausbleibenden Lawineneffekt in der Driftkammer
zurückführen.
Ab etwa \SI{2.5}{\kilo\volt} hebt sich das Signal mit der Probe deutlich vom
Untergrund ab. Der Signal-zu-Rausch-Abstand ist bei einer Spannung größer
\SI{2.8}{\kilo\volt} ausreichend für genauere Messungen.




\section{Driftzeitspektren}
Nun nehmen wir mit dem \emph{fpexperiment}-Programm
Driftzeitspektren bei verschiedenen Parametern auf.

\begin{figure}
    \centering
    \includegraphics[width=0.8 \linewidth]{m02_01_time_le_hist.pdf}
    \caption{%
	    Driftzeitspektren unter verschiedenen Hochspannungen
   }
    \label{fig:m02_01-driftzeitspektrum}
\end{figure}


\begin{figure}
    \centering
    \includegraphics[width=0.8 \linewidth]{m02_02_time_le_hist.pdf}
    \caption{%
	    Driftzeitspektren unter verschiedenen Verzögerungen
   }
    \label{fig:m02_03-driftzeitspektrum}
\end{figure}
\begin{figure}
    \centering
    \includegraphics[width=0.8 \linewidth]{m02_03_time_le_hist.pdf}
    \caption{%
	    Driftzeitspektren unter verschiedenen Diskriminatorschwellen
   }
    \label{fig:m02_03-driftzeitspektrum}
\end{figure}



%TODO verschieben der Probe > kaputter Szintilationszähler

\chapter{Auswertung der Langzeitmessung}

%TODO

\chapter{Ergebnis}

%TODO


%%%%%%%%%%%%%%%%%%%%%%%%%%%%%%%%%%%%%%%%%%%%%%%%%%%%%%%%%%%%%%%%%%%%%%%%%%%%%%%
%                                   Anhang                                    %
%%%%%%%%%%%%%%%%%%%%%%%%%%%%%%%%%%%%%%%%%%%%%%%%%%%%%%%%%%%%%%%%%%%%%%%%%%%%%%%

\begin{appendix}

%TODO

\end{appendix}

\end{document}

% vim: spell spelllang=de tw=79
