\documentclass[11pt, ngerman, fleqn, DIV=15, headinclude, BCOR=2cm]{scrreprt}

\usepackage{../../header}

\usepackage{placeins}
%\usepackage[maxfloats=50]{morefloats}

\usepackage{csquotes}

\usepackage{tikz}
\usetikzlibrary{chains}
\usetikzlibrary{shapes.geometric}

\tikzset{device/.style={
                rectangle,
                minimum size=6mm,
                draw=black
            },
            monitor/.style={
                rectangle,
                rounded corners=2mm,
                minimum size=6mm,
                draw=black
            },
        }

\usepackage{pgfplots}
\pgfplotsset{
    compat=1.9,
    width=0.8\linewidth,
    xticklabel style={/pgf/number format/use comma},
    yticklabel style={/pgf/number format/use comma},
}
\usepgfplotslibrary{polar}

\usepgfplotslibrary{external}
\tikzexternalize[mode=list and make]
\tikzsetexternalprefix{Abbildung-}

\DeclareSIUnit{\skt}{SKT}

\usepackage{booktabs}

\usepackage{subcaption}

\hypersetup{
    pdftitle=
}

\subject{Praktikumsprotokoll}
\title{Driftkammer}
\subtitle{Versuch P515 -- Universität Bonn}
\author{
	Frederike Schrödel \\
	\small{\href{mailto:fschroedel@gmx.de}{fschroedel@gmx.de}}
	\and
	Simon Schlepphorst \\
	\small{\href{mailto:s2@uni-bonn.de}{s2@uni-bonn.de}}
}

\date{2015-11-02}

\publishers{Tutor: Dr. Jürgen Hannappel
}

\begin{document}

\maketitle

\begin{abstract}
Ziel des Driftkammerversuches ist es, die Funktionsweise und Datengewinnung aus einer Driftkammer zu verstehen.
Hierzu nehmen wir dem Prototyp einer Driftkammer in Betrieb, 
bestimmen dessen Ansprechverhalten bei verschiedenen Hochspannungen mit oder ohne Bestrahlung durch ein $\beta$-Strahler  
und ermitteln seine Orts-Driftzeitbeziehung.
\end{abstract}

\tableofcontents

\chapter{Theorie}

\section{Einführung}
In der modernen Physik spielen Teilchenbeschleuniger eine wichtige Rolle und mit ihnen auch verschiedene Teilchendetektoren.
Wir beschäftigen uns im folgenden mit einer Driftkammer als Beispiel für ein gasbasierter Teilchendetektor.
Aufgabe der Driftkammer ist es den Ort eines bewegten Teilchen zu bestimmen.
In diesen Versuch werden die charaktristischen Eigenschaften des Prototypen einer Driftkammer des BGO-OD-Experimentes untersucht.
Hierbei wird zum einen das Verhalten der Driftkammer mit und ohne
$^90\text{Sr}$-Präperat unter verschiedenen Parametern verglichen.
Außerdem wird eine Langzeitmessung ohne Präperat durchgeführt, bei der die Winkelverteilung der kosmischen Strahlung vermessen wird.

\section{Theorie}
\subsection{Aufbau der Driftkammer}
Die Driftkammer besteht aus mit Agon-Kohlenstoffdioxidgemisch (im Verhältnis $ 82\%-18\% $ ) gefüllten Kammer, welche mit Drähten durchspannt ist. 
Hierbei bilden jeweils sechs hexagonal angeordnten Potentialdrähte, an denen
eine hohe negative spannung anliegt, und der in der Mitte liegede geerdete Anodendraht eine Driftzelle.
Die Potentialdrähte erfüllen hierbei die Aufgabe ein zylindrisches Feld zu erzeugen. 
Dieses soll die, durch Ionisierung frei werdenen Elektronen in die Richtung
des Anodendraht zu beschleunigen und verhindern, dass das Elektron in einer anderen Zelle detektiert wird.
Die Spannung der Potentialdrähte kann zwischen \SIrange{0}{3000}{\volt} variiert werden.

Ordnet man die einzelnen Driftzellen wie in Abbildung \fehlt an, so erhält man eine Driftkammer.
Da das Verhältnis der Gaszusammensetzung, dessen Druck und Temperatur
entscheidend für die Driftgeschwindigkeit der Elektronen ist, wird das Gasgemisch kontinuierlich getauscht. 
Die longitudinale Diffusion soll dabei möglichst klein gehalten werden,
anderenfalls wird die Ortsrekonstruktion ungenauer.
Das hat den Grund, dass die Longitudinale Diffusion die Zeitmessung beeinträchtigt.
Um durch die Driftgeschwindigkeit nicht durch die Bewegung des Gases stark zu verfälschen läuft dieser Vorgang sehr langsam ab.
% \fehlt bluber

\subsection{Ionisation der Gasatome}
Die Driftkammer ist der Prototyp zu einer Driftkammer die benutzt wird, um
die Bewegung schwerer geladener Teilchen zu detektieren.
Um den Energieverlust dieser sehr schnellen, geladenen Teilchen zu bestimmen kann man die Bethe-Bloch-Formel nutzen.
\[
    -\dod Ex = \frac{4 \pi r^2_e m_e c^2 N_\text{A} Z z^2}{A \beta^2}\sbr{\ln\sbr{\frac{2m_e c^2 \beta^2}{(1-\beta^2)I}}-\beta^2}
\]
Wobei $N_\text A$ die Avogadro-Zahl, $\beta c$ die Geschwindigkeit und $z$ die Ladung des Teilchens. $A$ und $Z$ sind die Massen- beziehungsweise Ordnungszahl, welche näherungsweise proportional ist zu dem Ionisationspotential $I$.

Die Abgabe der Energie findet hauptsächlich durch Ionisation statt.
Dies nennt man Primärionisation.
Die Anzahl der Primärionisationen hängt hierbei von dem Gas ab.
Bei diesen Vorgang frei werdene Energie ist größer als die Energie die für weitere Ionisationen gebraucht wird, wodurch es zu Sekundärionisationen kommt.

\subsection{Gasverstärkung, Signalentstehung und Ortsmessung}
Die freien Elektronen werden, wie oben beschrieben, aufgrund der Driftzellenengeomertie Richtung Anodendraht beschleunigt. 
Nahe des Anodendrahtes ist das elektische Feld so stark, dass die beschleunigten Elektronen weitere Gasatome ionisieren können.
So kommt es zu einen Lawineneffekt.
Das Signal wird so um einen Faktor von \numrange{e5}{e6} verstärkt.

Die Driftzeit ermittelt man, indem man die Zeitdifferenz zwischen den Durchgang durch den sehr schnell reagierenden Szintilators und dem in der Driftkammer entstandenen Signal misst.
Mit Hilfe der Orts-Driftzeitbeziehung, auch ODB genannt, lässt sich aus der Driftzeit der Ort des Teilchendurchgangs ermitteln.

\fehlt %TODO mittlere freie Weglänge
%TODO Ortsbestimmung

\chapter{Bestimmung der Betriebsparameter}

\section{Aufnahme analoger Größen}
Zunächst soll eine Hochspannung an der Driftkammer ermittelt werden, mit 
der sowohl die
Strahlung des $ ^{90}\text{Sr}$-Präperat, als auch die kosmische Strahlung
detektiert werden kann.
Hierzu wird der Kammerstrom  in Abhängigkeit von der Hochspannung, die an den
Potentialdrähten anliegt, gemessen. 
Dies wird dadurch realisiert, dass die über einem \SI{1}{\mega\ohm} Widerstand
abfallende  Spannung aus der Driftkammer gemessen wird.
Da die Kabel und die Driftkammer eine Kapazität aufweisen, muss nach jeder Änderung der
Hochspannung gewartet werde, bis die Spannung sich auf einen Wert eingestellt
hat, bevor die Messung beginnen kann.

\begin{figure}
	\begin{subfigure}{0.49 \linewidth}
        \centering
        \includegraphics[width=\linewidth]{m01_01_151026.pdf}
        \caption{%
		Datenpunkte mit Fehler
       }
        \label{fig:m01_01_messdaten}
    \end{subfigure}
    \begin{subfigure}{0.49 \linewidth}
        \centering
        \includegraphics[width=\linewidth]{m01_01_151026_2.pdf}
        \caption{%
		Relevanz der Messung
	}
        \label{fig:m01_01_relevanz}
    \end{subfigure}
    \caption{%
	    Messung des Kammerstroms abhängig von der Anodenspannung bei mit
	    Atmosphäre gefluteter Kammer
    }
    \label{fig:m01_01_plots}
\end{figure}


Bei dem Vergleich der beiden Kennlinien, Kammerstrom gegen Hochspannung,
des Versuchs mit und ohne Bestrahlung fiel auf, dass es zunächst keinen
erkennbaren unterschied der Kennlinien gab.
Nach Überprüfung des Aufbaus, stellte sich raus, dass die Kammer nicht mit dem 
gewünschten Gasgemisch gefüllt war.

Nach dem die Kammer über Nacht mit dem Gasgemisch gespült wurde, lieferte der
Versuch dirket sinnvolle Ergebnisse.

\begin{figure}
	\centering
	\begin{subfigure}{0.49 \linewidth}
        \includegraphics[width=\linewidth]{m01_02_151027.pdf}
        \caption{%
		Datenpunkte mit Fehler
        }
        \label{fig:m01_02_messdaten}
    \end{subfigure}
    \begin{subfigure}{0.49 \linewidth}
        \includegraphics[width=\linewidth]{m01_02_151027_2.pdf}
        \caption{%
		Relevanz der Messung
       }
        \label{fig:m01_02_relevanz}
    \end{subfigure}
    \caption{%
	    Messung des Kammerstroms abhängig von der Anodenspannung bei mit
	    $82/18$ Ag/C Gemisch gefluteter Kammer
    }
    \label{fig:m01_02_plots}
\end{figure}

Bei niedrigen Spannungen lassen sich die Ereignisse nicht vom Hintergrund
trennen. 
Dies lässt sich auf den ausbleibenden Lawineneffekt in der Driftkammer
zurückführen.
Für hohe Spannungen lässt sich auch für den Hintergrund ein leichter
Anstieg des Anodenstroms erkennen, allerdings bleibt dieser deutlich hinter der
Probe zurück.
Ab etwa \SI{2.5}{\kilo\volt} hebt sich das Signal mit der Probe deutlich vom
Untergrund ab. Der Signal-zu-Rausch-Abstand ist bei einer Spannung größer
\SI{2.8}{\kilo\volt} ausreichend für genauere Messungen.

\section{Driftzeitspektren}
Nun wurden mit der $^{90}\text{Sr}$ Quelle und dem \emph{fpexperiment}-Programm
Driftzeitspektren bei verschiedenen Parametern aufgenommen.
Variiert wurden dabei nacheinander die Hochspannung, die Verzögerung und die
Diskriminatorschwelle.
Ziel ist es eine möglichst sinnvolle Einstellung zu finden, mit der die
Langzeitmessung aufgenommen wird.

Der Szintilationszähler wurde unterhalbt der Driftkammer und orthogonal zu
den Drähten positioniert.
Dabei fiel auf, dass der Szintilationszähler näher an seiner Basis um
Größenordnungen  besser ansprach.
Die folgenden Messreihen wurden deshalb mit der Probe am Rand der
Driftkammer aufgenommen.

Bei dem erstellen der Plots fiel auf, dass die Draht Nummerierung paarweise
vertauscht war.
Nach Anpassung der Nummerierung erhielten wir folgende Plots bei denen der Übersichtlichkeit halber auf Fehlerbalken
verzichtet wurde. 
Bei der Korrelation der Ansprecher wird dies besonders deutlich.

Als erstes haben wir Driftzeitspektren für verschiedene Hochspannungen
aufgenommen.
Als Hochspannungen haben wir \SI{2,86}{\kilo\volt},
\SI{2,9}{\kilo\volt}, \SI{2,95}{\kilo\volt} und \SI{3,021}{\kilo\volt}.
Im Vergleich sieht man, dass wir uns mit den gewählten Spannungen in einen
Bereich bewegen, indem es kaum Unterschiede gibt.
Deshalb haben wir uns für die weiteren Messungen für den mittleren Wert von
\SI{2,9}{\kilo\volt} entschieden.
Gut deutlich wird allerdings, dass man für kurze Driftzeiten die meisten
Ansprecher erhält. Da dies die Ansprecher für den Durchgang nahe des
Anodendrahtes, also im Bereich der hohen Feldstärke ist, ist es sinnvoll das
diese Ereignisse bevorzugt detektiert werden.
In diesen Bereich ist die Wahrscheinlichkeit eine Lawine auszulösen
besonders hoch.
Das zweite Maximum verschiebt sich mit steigender Spannung
weiter Richtung kürzerer Driftzeit. Das lässt sich damit begründen, dass das
Feld ansteigt und dadurch die Elektronen stärker beschleunigt werden.

\begin{figure}
    \centering
    \includegraphics[width=0.8 \linewidth]{m02_01_time_le_hist.pdf}
    \caption{%
	    Driftzeitspektren unter verschiedenen Hochspannungen
   }
    \label{fig:m02_01_driftzeitspektrum}
\end{figure}

Als nächstes betrachten wir den Einfluss der unterschiedlichen Verzögerungen.
Da wir wollen, dass das Driftzeitspektrum bei einer Driftzeit von
\SI{100}{\nano\second} beginnt haben wir uns für die Verzögerung von
$0x0028$, also das Rote, entschieden.

\begin{figure}
    \centering
    \includegraphics[width=0.8 \linewidth]{m02_02_time_le_hist.pdf}
    \caption{%
	    Driftzeitspektren unter verschiedenen Verzögerungen
   }
    \label{fig:m02_02_driftzeitspektrum}
\end{figure}

%TODO Wenn man Diskriminatorschwelle ändert, dann

\begin{figure}
    \centering
    \includegraphics[width=0.8 \linewidth]{m02_03_time_le_hist.pdf}
    \caption{%
	    Driftzeitspektren unter verschiedenen Diskriminatorschwellen
   }
    \label{fig:m02_03_driftzeitspektrum}
\end{figure}



%TODO verschieben der Probe > kaputter Szintilationszähler

\clearpage
\chapter{Auswertung der Langzeitmessung}

Mit den im letzten Kapitel ermittelten Parametern wurde eine Langzeitmessung
der Höhenstrahlung durchgeführt.

Der Szintilationszähler wurde dazu parallel zu den Drähten oberhalb der
Driftkammer positioniert.

Um das immer noch vorhandene Hintergrundrauschen von den deutlichen
Ereignissen zu trennen wurden alle Ereignisse mit einer \emph{Time over
Treshold}$ < 60 \cdot \SI{2.5}{\nano\second}$ verworfen. 
Die Ansprecher auf den einzelnen Drähten sind um die Position des
Szintilationszählers auf der Driftkammer verteilt.


\begin{figure}
	\centering
	\begin{subfigure}{0.49 \linewidth}
		\includegraphics[width=\linewidth]{m03_01_tot_vs_time.pdf}
		\caption{%
			ungefiltert
		}
		\label{fig:m03_tot_vs_time_unfiltered}
	\end{subfigure}
	\begin{subfigure}{0.49 \linewidth}
		\includegraphics[width=\linewidth]{m03_01_tot_vs_time_filtered.pdf}
		\caption{%
			nach Time over Treshold gefiltert
		}
		\label{fig:m03_tot_vs_time_filtered}
	\end{subfigure}
	\caption{%
		Langszeitmessung: Time over Treshold / Driftzeit Histogramme
	}
	\label{fig:m03_tot_vs_time}
\end{figure}

\begin{figure}
	\centering
	\begin{subfigure}{0.49 \linewidth}
		\includegraphics[width=\linewidth]{m03_01_wire_le_hist_filtered.pdf}
		\caption{%
			Verteilung der Ansprecher auf die Drähte
		}
		\label{fig:m03_wire_le_hist_filtered}
	\end{subfigure}
	\begin{subfigure}{0.49 \linewidth}
		\includegraphics[width=\linewidth]{m03_01_time_vs_wire_filtered.pdf}
		\caption{%
			Verteilung der Driftzeiten auf die Drähte
		}
		\label{fig:m03_time_vs_wire_filtered}
	\end{subfigure}
	\caption{%
		Langzeitmessung: Verteilung der Ansprecher und Driftzeiten in der Kammer
	}
	\label{fig:m03_verteilung_in_der_kammer}
\end{figure}




\begin{figure}
	\centering
	\includegraphics[width=\linewidth]{m03_01_wire_correlation_filtered.pdf}
	\caption{%
		Langzeitmessung: Korrelation benachbarter Drähte
	}
	\label{fig:m03_wire_correlation}
\end{figure}

An der Korrelation der benachbarten Drähte ist zu sehen, dass mit zunehmendem
Abstand zu Draht 34, nicht mehr die unmittelbaren Nachbarn korrelieren. Das
lässt auf einen zunehmenden Winkel schließen, unter dem die ionisierenden
Teilchen die Driftkammer durchquerten, nachdem sie den Szintilationszähler
ausgelößt haben.


Anschließend wurde aus dem Driftzeitspektrum der verbliebenden Daten die ODB bestimmt


\begin{figure}
	\begin{subfigure}{0.49 \linewidth}
		\centering
		\includegraphics[width=\linewidth]{m03_01_time_le_hist_filtered.pdf}
		\caption{%
			Driftzeitspektrum
		}
		\label{fig:m03_time_le_hist_filtered}
	\end{subfigure}
	\begin{subfigure}{0.49 \linewidth}
		\centering
		\includegraphics[width=\linewidth]{m03_01_odb_filtered.pdf}
		\caption{%
			ODB: Aufintegriertes Driftzeitspektrum
		}
		\label{fig:m03_odb_filtered}
	\end{subfigure}
	\caption{%
		Langszeitmessung: Bestimmung der ODB aus dem Driftzeitspektrum
	}
	\label{fig:m03_odb_bestimmung}
\end{figure}


\begin{figure}
	\centering
	\includegraphics[width=\linewidth]{m03_01_summe_filtered.pdf}
	\caption{%
		Langszeitmessung: Abstandssumme gegen Abstandsdifferenz
	}
	\label{fig:m03_ort_driftzeit}
\end{figure}

Mit der Abstandsbeziehung aus der ODB wurden nun die Abstandssummen und
Abstandsdifferenzen benachbarter Drahtpaare in einem 2D Histogramm
aufgetragen.
Hier zeigt sich sehr neben einem starken Hintergrund auch ein
ausgewachsenens Binningproblem im Graphen selbst.


%TODO mehr Inhalte

\chapter{Ergebnis}

%TODO


%%%%%%%%%%%%%%%%%%%%%%%%%%%%%%%%%%%%%%%%%%%%%%%%%%%%%%%%%%%%%%%%%%%%%%%%%%%%%%%
%                                   Anhang                                    %
%%%%%%%%%%%%%%%%%%%%%%%%%%%%%%%%%%%%%%%%%%%%%%%%%%%%%%%%%%%%%%%%%%%%%%%%%%%%%%%

\begin{appendix}

	\chapter{Graphen zur Bestimmung der Betriebsparameter}

	%m02.01

	%Ansprecher / Kabel Histogramme
	\begin{figure}
		\centering
	\begin{subfigure}[a]{0.45 \textwidth}
		\includegraphics[width=\textwidth]{m02_01_run_2860V_wire_le_hist.pdf}
		\caption{%
		}
		\label{fig:m02_01_run_2860V_wire_le_hist}
	\end{subfigure}
	\begin{subfigure}[a]{0.45 \textwidth}
		\includegraphics[width=\textwidth]{m02_01_run_2900V_wire_le_hist.pdf}
		\caption{%
		}
		\label{fig:m02_01_run_2900V_wire_le_hist}
	\end{subfigure}\\
	\begin{subfigure}[a]{0.45 \textwidth}
		\includegraphics[width=\textwidth]{m02_01_run_2950V_wire_le_hist.pdf}
		\caption{%
		}
		\label{fig:m02_01_run_2950V_wire_le_hist}
	\end{subfigure}
	\begin{subfigure}[a]{0.45 \textwidth}
		\includegraphics[width=\textwidth]{m02_01_run_3021V_wire_le_hist.pdf}
		\caption{%
		}
		\label{fig:m02_01_run_3021V_wire_le_hist}
	\end{subfigure}
	\caption{%
	}
	\label{fig:m02_01_wire_le_hist}
	\end{figure}


	%Driftzeit / Kabel Histogramme
	\begin{figure}
		\centering
	\begin{subfigure}[a]{0.45 \textwidth}
		\includegraphics[width=\textwidth]{m02_01_run_2860V_time_vs_wire.pdf}
		\caption{%
		}
		\label{fig:m02_01_run_2860V_time_vs_wire}
	\end{subfigure}
	\begin{subfigure}[a]{0.45 \textwidth}
		\includegraphics[width=\textwidth]{m02_01_run_2900V_time_vs_wire.pdf}
		\caption{%
		}
		\label{fig:m02_01_run_2900V_time_vs_wire}
	\end{subfigure}
	\\
	\begin{subfigure}[a]{0.45 \textwidth}
		\includegraphics[width=\textwidth]{m02_01_run_2950V_time_vs_wire.pdf}
		\caption{%
		}
		\label{fig:m02_01_run_2950V_time_vs_wire}
	\end{subfigure}
	\begin{subfigure}[a]{0.45 \textwidth}
		\includegraphics[width=\textwidth]{m02_01_run_3021V_time_vs_wire.pdf}
		\caption{%
		}
		\label{fig:m02_01_run_3021V_time_vs_wire}
	\end{subfigure}
	\caption{%
	}
	\label{fig:m02_01_time_vs_wire}
	\end{figure}



	%Time over Treshold / Driftzeit  Histogramme
	\begin{figure}
		\centering
	\begin{subfigure}[a]{0.45 \textwidth}
		\includegraphics[width=\textwidth]{m02_01_run_2860V_tot_vs_time.pdf}
		\caption{%
		}
		\label{fig:m02_01_run_2860V_tot_vs_time}
	\end{subfigure}
	\begin{subfigure}[a]{0.45 \textwidth}
		\includegraphics[width=\textwidth]{m02_01_run_2900V_tot_vs_time.pdf}
		\caption{%
		}
		\label{fig:m02_01_run_2900V_tot_vs_time}
	\end{subfigure}
	\begin{subfigure}[a]{0.45 \textwidth}
		\includegraphics[width=\textwidth]{m02_01_run_2950V_tot_vs_time.pdf}
		\caption{%
		}
		\label{fig:m02_01_run_2950V_tot_vs_time}
	\end{subfigure}
	\begin{subfigure}[a]{0.45 \textwidth}
		\includegraphics[width=\textwidth]{m02_01_run_3021V_tot_vs_time.pdf}
		\caption{%
		}
		\label{fig:m02_01_run_3021V_tot_vs_time}
	\end{subfigure}
	\caption{%
	}
	\label{fig:m02_01_tot_vs_time}
	\end{figure}

	\clearpage


	%m02.02

	%Ansprecher / Kabel Histogramme
	\begin{figure}
		\centering
	\begin{subfigure}[a]{0.45 \textwidth}
		\includegraphics[width=\textwidth]{m02_02_run_0x0020_wire_le_hist.pdf}
		\caption{%
		}
		\label{fig:m02_02_run_0x0020_wire_le_hist}
	\end{subfigure}
	\begin{subfigure}[a]{0.45 \textwidth}
		\includegraphics[width=\textwidth]{m02_02_run_0x0024_wire_le_hist.pdf}
		\caption{%
		}
		\label{fig:m02_02_run_0x0024_wire_le_hist}
	\end{subfigure}
	\begin{subfigure}[a]{0.45 \textwidth}
		\includegraphics[width=\textwidth]{m02_02_run_0x0028_wire_le_hist.pdf}
		\caption{%
		}
		\label{fig:m02_02_run_0x0028_wire_le_hist}
	\end{subfigure}
	\begin{subfigure}[a]{0.45 \textwidth}
		\includegraphics[width=\textwidth]{m02_02_run_0x0038_wire_le_hist.pdf}
		\caption{%
		}
		\label{fig:m02_02_run_0x0038_wire_le_hist}
	\end{subfigure}
	\caption{%
	}
	\label{fig:m02_02_wire_le_hist}
	\end{figure}


	%Driftzeit / Kabel Histogramme
	\begin{figure}
		\centering
	\begin{subfigure}[a]{0.45 \textwidth}
		\includegraphics[width=\textwidth]{m02_02_run_0x0020_time_vs_wire.pdf}
		\caption{%
		}
		\label{fig:m02_02_run_0x0020_time_vs_wire}
	\end{subfigure}
	\begin{subfigure}[a]{0.45 \textwidth}
		\includegraphics[width=\textwidth]{m02_02_run_0x0024_time_vs_wire.pdf}
		\caption{%
		}
		\label{fig:m02_02_run_0x0024_time_vs_wire}
	\end{subfigure}
	\begin{subfigure}[a]{0.45 \textwidth}
		\includegraphics[width=\textwidth]{m02_02_run_0x0028_time_vs_wire.pdf}
		\caption{%
		}
		\label{fig:m02_02_run_0x0028_time_vs_wire}
	\end{subfigure}
	\begin{subfigure}[a]{0.45 \textwidth}
		\includegraphics[width=\textwidth]{m02_02_run_0x0038_time_vs_wire.pdf}
		\caption{%
		}
		\label{fig:m02_02_run_0x0038_time_vs_wire}
	\end{subfigure}
	\caption{%
	}
	\label{fig:m02_02_time_vs_wire}
	\end{figure}

	%Time over Treshold / Driftzeit  Histogramme
	\begin{figure}
		\centering
	\begin{subfigure}[a]{0.45 \textwidth}
		\includegraphics[width=\textwidth]{m02_02_run_0x0020_tot_vs_time.pdf}
		\caption{%
		}
		\label{fig:m02_02_run_0x0020_tot_vs_time}
	\end{subfigure}
	\begin{subfigure}[a]{0.45 \textwidth}
		\includegraphics[width=\textwidth]{m02_02_run_0x0024_tot_vs_time.pdf}
		\caption{%
		}
		\label{fig:m02_02_run_0x0024_tot_vs_time}
	\end{subfigure}
	\begin{subfigure}[a]{0.45 \textwidth}
		\includegraphics[width=\textwidth]{m02_02_run_0x0028_tot_vs_time.pdf}
		\caption{%
		}
		\label{fig:m02_02_run_0x0028_tot_vs_time}
	\end{subfigure}
	\begin{subfigure}[a]{0.45 \textwidth}
		\includegraphics[width=\textwidth]{m02_02_run_0x0038_tot_vs_time.pdf}
		\caption{%
		}
		\label{fig:m02_02_run_0x0038_tot_vs_time}
	\end{subfigure}
	\caption{%
	}
	\label{fig:m02_02_tot_vs_time}
	\end{figure}
	\clearpage


	%m02.03

	%Ansprecher / Kabel Histogramme
	\begin{figure}
		\centering
	\begin{subfigure}[a]{0.45 \textwidth}
		\includegraphics[width=\textwidth]{m02_03_run_0x0020_wire_le_hist.pdf}
		\caption{%
		}
		\label{fig:m02_03_run_0x0020_wire_le_hist}
	\end{subfigure}
	\begin{subfigure}[a]{0.45 \textwidth}
		\centering
		\includegraphics[width=\textwidth]{m02_03_run_0x002D_wire_le_hist.pdf}
		\caption{%
		}
		\label{fig:m02_03_run_0x002D_wire_le_hist}
	\end{subfigure}
	\begin{subfigure}[a]{0.45 \textwidth}
		\centering
		\includegraphics[width=\textwidth]{m02_03_run_0x0040_wire_le_hist.pdf}
		\caption{%
		}
		\label{fig:m02_03_run_0x0040_wire_le_hist}
	\end{subfigure}
	\caption{%
	}
	\label{fig:m02_03_wire_le_hist}
	\end{figure}


	%Driftzeit / Kabel Histogramme
	\begin{figure}
		\centering
	\begin{subfigure}[a]{0.45 \textwidth}
		\includegraphics[width=\textwidth]{m02_03_run_0x0020_time_vs_wire.pdf}
		\caption{%
		}
		\label{fig:m02_03_run_0x0020_time_vs_wire}
	\end{subfigure}
	\begin{subfigure}[a]{0.45 \textwidth}
		\includegraphics[width=\textwidth]{m02_03_run_0x002D_time_vs_wire.pdf}
		\caption{%
		}
		\label{fig:m02_03_run_0x002D_time_vs_wire}
	\end{subfigure}
	\begin{subfigure}[a]{0.45 \textwidth}
		\includegraphics[width=\textwidth]{m02_03_run_0x0040_time_vs_wire.pdf}
		\caption{%
		}
		\label{fig:m02_03_run_0x0040_time_vs_wire}
	\end{subfigure}
	\caption{%
	}
	\label{fig:m02_03_time_vs_wire}
	\end{figure}


	%Time over Treshold / Driftzeit  Histogramme
	\begin{figure}
		\centering
	\begin{subfigure}[a]{0.45 \textwidth}
		\includegraphics[width=\textwidth]{m02_03_run_0x0020_tot_vs_time.pdf}
		\caption{%
		}
		\label{fig:m02_03_run_0x0020_tot_vs_time}
	\end{subfigure}
	\begin{subfigure}[a]{0.45 \textwidth}
		\includegraphics[width=\textwidth]{m02_03_run_0x002D_tot_vs_time.pdf}
		\caption{%
		}
		\label{fig:m02_03_run_0x002D_tot_vs_time}
	\end{subfigure}
	\begin{subfigure}[a]{0.45 \textwidth}
		\includegraphics[width=\textwidth]{m02_03_run_0x0040_tot_vs_time.pdf}
		\caption{%
		}
		\label{fig:m02_03_run_0x0040_tot_vs_time}
	\end{subfigure}
	\caption{%
	}
	\label{fig:m02_03_tot_vs_time}
	\end{figure}




%TODO

\end{appendix}
\end{document}

% vim: spell spelllang=de tw=79
