\documentclass[11pt, ngerman, fleqn, DIV=15, headinclude, BCOR=2cm]{scrreprt}

\usepackage{../../header}

\usepackage{placeins}
%\usepackage[maxfloats=50]{morefloats}

\usepackage{csquotes}

\usepackage{tikz}
\usetikzlibrary{chains}
\usetikzlibrary{shapes.geometric}

\tikzset{device/.style={
                rectangle,
                minimum size=6mm,
                draw=black
            },
            monitor/.style={
                rectangle,
                rounded corners=2mm,
                minimum size=6mm,
                draw=black
            },
        }

\usepackage{pgfplots}
\pgfplotsset{
    compat=1.9,
    width=0.8\linewidth,
    xticklabel style={/pgf/number format/use comma},
    yticklabel style={/pgf/number format/use comma},
}
\usepgfplotslibrary{polar}

\usepgfplotslibrary{external}
\tikzexternalize[mode=list and make]
\tikzsetexternalprefix{Abbildung-}

\DeclareSIUnit{\skt}{SKT}

\usepackage{booktabs}

\hypersetup{
    pdftitle=
}

\newcommand{\plotwidth}{0.8\linewidth}

\subject{Praktikumsprotokoll}
\title{$\gamma$-Spektroskopie mit Szintillations- und Halbleiterdetektoren}
\subtitle{Versuch P521 -- Universität Bonn}
\author{
	Frederike Schrödel \\
	\small{\href{mailto:fschroedel@gmx.de}{fschroedel@gmx.de}}
	\and
	Simon Schlepphorst \\
	\small{\href{mailto:s2@uni-bonn.de}{s2@uni-bonn.de}}
}

\date{2015-11-10}

\publishers{Tutor: %TODO
}

\begin{document}

\maketitle

\begin{abstract}
In diesen Versuch geht es darum die Eigenschaften von einem
Szintillationsdetektor und einem Ge-Halbleiterdetektor bei der
$\gamma$-Spektroskopie zu vergleiche.
Dabei sind die Energieauflösung und die Nachweiswahrscheinlichkeit die
zu untersuchenden Eigenschaften.
\end{abstract}


\tableofcontents

\chapter{Theorie}

\section{Wechselwirkung von $\gamma$-Strahlung mit Materie}
In der Materie kann die $gamma$-Strahlung auf verschiedene Weise mit der
Materie wechselwirken.

\subsection{Photoeffekt}
Der Photoeffekt beschreibt den Vorgang, wie ein Photon ein Elektron aus dem
Atom auslöst. 
Hierbei gibt ein Elektron seine komplette Energie ab.
Die kinetische Energie des ausgelösten Elektrons hängt nur über die
Bindungsenergie ($E_\text{Bind.}$) mit dem der Energie des Photons ($h\nu$), also dessen Frequenz ($\nu$) zusammen.
Dieser Prozess kann nur in der nähe eines Atomkerns statt finden, weil dieser
den Rückstoß aufnimmt.
So bleibt bei diesen Vorgang Energie und Impuls erhalten. 
\[ 
    E = h\nu - E_\text{Bind.}
\]
Wenn man sich den Wirkungsquerschnitt $\sigma$ für diesen Prozess anschaut,
dann stellt man eine große Abhängigkeit von der Ordnungszahl $Z$ fest.
\[
    \sigma \propto Z^5 E_\gamma^{\frac 72}
\]
Deshalb kommt dieser Effekt besonders häufig in Stoffen mit hoher Ordnungszahl
vor.

\subsection{Paarerzeugung}
Ein weiterer Effekt der bei Energien $E_\gamma \ge 2m_\text ec^2$ auftreten
kann, ist die Elektron- Positron Paarbildung.
Dieser Effekt ist allerdings erst ab einer Energie von
\SI{}{\mega\electronvolt} relevant.
% Für den Wirkungsquerschnitt gilt $\sigma_\text{Pc} \propto Z^2\ln(E_\gamma)$

\subsection{Comptonstreuung}
Wenn Photonen an einem Elektron streuen, dann spricht man von dem
Comptoneffekt. 
Hierbei gibt das Photon ein Teil seiner an das Elektron ab und ändert dabei
seine Frequenz.
Wie viel seiner Energie das Photon abgibt hängt dabei vom Winkel ab.
Für die geänderte Energie gilt:
\[
    E_\nu'(\phi) = \frac{E_\nu}{1+\frac{E_\nu}{m_\text ec^2}(1+\cos(\phi))}
\]
Die größte Energieänderung erhält man also bei einem Winkel von
$\phi=\SI{180}{\degree}$.
Der Wirkungsquerschnitt entspricht $\sigma_\text C \propto \frac{Z}{E_\gamma}$.

\section{Szintillationdetektor}
Mit einem Szintillationsdetektor lassen sich die Intensität und Energie von
ionisierender Strahlung messen.
Er besteht aus einem Szintilator und einem Photomultiplier, der das Signal
verstärkt.


\chapter{Durchführung}

%TODO

\chapter{Auswertung}

%TODO

\begin{figure}
    \centering
    \includegraphics[width=\plotwidth]{Ge_calib}
    \caption{%
        %
    }
    \label{fig:}
\end{figure}

\begin{figure}
    \centering
    \includegraphics[width=\plotwidth]{Ge_calib-peaks}
    \caption{%
        %
    }
    \label{fig:}
\end{figure}

\begin{tabular}{SSSSSS}
    {Nummer} & {Kanal} & {Breite $\Gamma$} & {Fläche} & {rel.\ Intens.} &
    {$E_\gamma$ / \si{\kilo\electronvolt}} \\
    \midrule
    %< for row in ge_co_fits_table >%
    << ' & '.join(row) >> \\
    %< endfor >%
\end{tabular}


Germanium
\[
    \text{Energie} =
    \text{Kanalnummer} \cdot \SI{<< ge_slope >>}{\kilo\electronvolt}
    +
    \SI{<< ge_offset >>}{\kilo\electronvolt} \,.
\]

\begin{figure}
    \centering
    \includegraphics[width=\plotwidth]{ge_channels}
    \caption{%
        %
    }
    \label{fig:}
\end{figure}

Szintillationszähler
\[
    \text{Energie} =
    \text{Kanalnummer} \cdot \SI{<< sz_slope >>}{\kilo\electronvolt}
    +
    \SI{<< sz_offset >>}{\kilo\electronvolt} \,.
\]

\begin{figure}
    \centering
    \includegraphics[width=\plotwidth]{Sz-CO-peaks}
    \caption{%
        %
    }
    \label{fig:}
\end{figure}

\begin{figure}
    \centering
    \includegraphics[width=\plotwidth]{Sz-CS-peaks}
    \caption{%
        %
    }
    \label{fig:}
\end{figure}

\begin{figure}
    \centering
    \includegraphics[width=\plotwidth]{Sz-EU-peaks}
    \caption{%
        %
    }
    \label{fig:}
\end{figure}

\begin{figure}
    \centering
    \includegraphics[width=\plotwidth]{sz_channels}
    \caption{%
        %
    }
    \label{fig:}
\end{figure}

\[
    \text{Const} = \num{<< ge_const >>}\,\sqrt{\si{\kilo\electronvolt}}
    ,\qquad
    \Delta E_\mathrm e = \SI{<< ge_electronic_width >>}{\kilo\electronvolt}
\]

\begin{figure}
    \centering
    \includegraphics[width=\plotwidth]{halbwertsbreite_ge}
    \caption{%
        %
    }
    \label{fig:}
\end{figure}


\begin{figure}
    \centering
    \includegraphics[width=\plotwidth]{ge_efficiency}
    \caption{%
        %
    }
    \label{fig:}
\end{figure}

\begin{tabular}{SSSS}
    {Energie / \si{\kilo\electronvolt}} & {rel.\ Intens. erw.} & {rel.\ Intens.
mess.} & {Effizienz} \\
    \midrule
    %< for row in ge_efficiency_table >%
    << ' & '.join(row) >> \\
    %< endfor >%
\end{tabular}

\chapter{Ergebnis}

%TODO


%%%%%%%%%%%%%%%%%%%%%%%%%%%%%%%%%%%%%%%%%%%%%%%%%%%%%%%%%%%%%%%%%%%%%%%%%%%%%%%
%                                   Anhang                                    %
%%%%%%%%%%%%%%%%%%%%%%%%%%%%%%%%%%%%%%%%%%%%%%%%%%%%%%%%%%%%%%%%%%%%%%%%%%%%%%%

\begin{appendix}

%TODO

\end{appendix}

\end{document}

% vim: spell spelllang=de tw=79
