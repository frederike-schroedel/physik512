\documentclass[11pt, ngerman, fleqn, DIV=15, headinclude, BCOR=2cm]{scrreprt}

\usepackage{../../header}

\usepackage{placeins}
%\usepackage[maxfloats=50]{morefloats}

\usepackage{csquotes}

\usepackage{tikz}
\usetikzlibrary{chains}
\usetikzlibrary{shapes.geometric}

\tikzset{device/.style={
                rectangle,
                minimum size=6mm,
                draw=black
            },
            monitor/.style={
                rectangle,
                rounded corners=2mm,
                minimum size=6mm,
                draw=black
            },
        }

\usepackage{pgfplots}
\pgfplotsset{
    compat=1.9,
    width=0.8\linewidth,
    xticklabel style={/pgf/number format/use comma},
    yticklabel style={/pgf/number format/use comma},
}
\usepgfplotslibrary{polar}

\usepgfplotslibrary{external}
\tikzexternalize[mode=list and make]
\tikzsetexternalprefix{Abbildung-}

\DeclareSIUnit{\skt}{SKT}

\usepackage{booktabs}

\hypersetup{
    pdftitle=
}

\newcommand{\plotwidth}{0.8\linewidth}

\subject{Praktikumsprotokoll}
\title{$\gamma$-Spektroskopie mit Szintillations- und Halbleiterdetektoren}
\subtitle{Versuch P521 -- Universität Bonn}
\author{
	Frederike Schrödel \\
	\small{\href{mailto:fschroedel@gmx.de}{fschroedel@gmx.de}}
	\and
	Simon Schlepphorst \\
	\small{\href{mailto:s2@uni-bonn.de}{s2@uni-bonn.de}}
}

\date{2015-11-10}

\publishers{Tutor: %TODO
}

\begin{document}

\maketitle

\begin{abstract}
In diesen Versuch geht es darum die Eigenschaften von einem
Szintillationsdetektor und einem Ge-Halbleiterdetektor bei der
$\gamma$-Spektroskopie zu vergleiche.
Dabei sind die Energieauflösung und die Nachweiswahrscheinlichkeit die
zu untersuchenden Eigenschaften.
\end{abstract}


\tableofcontents

\chapter{Theorie}

\section{Wechselwirkung von $\gamma$-Strahlung mit Materie}
In der Materie kann die $\gamma$-Strahlung auf verschiedene Weise mit der
Materie wechselwirken. 
Diese Wechselwirkung hängt von der Ordnungszahl und der Energie der
$\gamma$-Strahlung ab.

\subsection{Photoeffekt}
Der Photoeffekt beschreibt den Vorgang, wie ein Photon ein Elektron aus dem
Atom auslöst. 
Hierbei gibt ein Elektron seine komplette Energie ab.
Die kinetische Energie des ausgelösten Elektrons hängt nur über die
Bindungsenergie ($E_\text{Bind.}$) mit dem der Energie des Photons ($h\nu$), also dessen Frequenz ($\nu$) zusammen.
Dieser Prozess kann nur in der nähe eines Atomkerns statt finden, weil dieser
den Rückstoß aufnimmt.
So bleibt bei diesen Vorgang Energie und Impuls erhalten. 
\[ 
    E = h\nu - E_\text{Bind.}
\]
Wenn man sich den Wirkungsquerschnitt $\sigma$ für diesen Prozess anschaut,
dann stellt man eine große Abhängigkeit von der Ordnungszahl $Z$ fest.
\[
    \sigma \propto Z^5 E_\gamma^{\frac 72}
\]
Deshalb kommt dieser Effekt besonders häufig in Stoffen mit hoher Ordnungszahl
vor.

\subsection{Paarerzeugung}
Ein weiterer Effekt der bei Energien $E_\gamma \ge 2m_\text ec^2$ auftreten
kann, ist die Elektron- Positron Paarbildung.
Dieser Effekt ist allerdings erst ab einer Energie von $E_\gamma >> 2m_\text
ec^2$ relevant.
Für den Wirkungsquerschnitt gilt $\sigma \propto Z^2$

\subsection{Comptonstreuung}
Wenn Photonen an einem Elektron streuen, dann spricht man von dem
Comptoneffekt. 
Hierbei gibt das Photon ein Teil seiner an das Elektron ab und ändert dabei
seine Frequenz.
Wie viel seiner Energie das Photon abgibt hängt dabei vom Winkel ab.
Für die geänderte Energie gilt:
\[
    E_\nu'(\phi) = \frac{E_\nu}{1+\frac{E_\nu}{m_\text ec^2}(1+\cos(\phi))}
\]
Die größte Energieänderung erhält man also bei einem Winkel von
$\phi=\SI{180}{\degree}$.
Diese Energieänderung berechnet sich durch:
\[
    \delta\lambda = \frac{h}{m_\text ec}(1-\cos(\phi))
\]
Der Wirkungsquerschnitt entspricht $\sigma_\text C \propto \frac{Z}{E_\gamma}$.

\section{Szintillationdetektor}
Mit einem Szintillationsdetektor lassen sich die Intensität und Energie von
ionisierender Strahlung messen.
Er besteht aus einem Szintillator und einem Photomultiplier, der das Signal
verstärkt.

Innerhalb das von Lichteinfall geschützten Szintillators entstehen durch
ionisierende Strahlung Lichtblitze. 
Wie viele dieser Blitze auftreten ist von der Energie der Strahlung abhängig.
Aufgrund des Photoeffektes, werden an der Photokathode Elektronen ausgelöst und
das so entstandene Signal im Photomultiplier verstärkt.
Durch den Aufbau des Photomultiplieres kommt es zu einem Lawineneffekt, welcher
die Elektronen verstärkt und somit eine Detektion ermöglicht.
Die Amplitude des so entstanden Strompulses ist somit auch proportional zu der
Energie der Eingangsstrahlung.

Wenn man eine gute Energieauflösung möchte, sollte man allerdings eher einem
Halbleiterdetektor nutzen.

\section{Halbleiterdetektor}
Die Basis des Halbleiterdetektors bildet die verarmungs Zone die zwischen einem
$p$ und einem $n$ dotieren Halbleiter entsteht.

Betrachtet man das Bändermodel vom Halbleitern, so kann man sich die Dotierung
des Halbleiters als weiteres dünnes Band knapp unter dem Leitungsband
($n$-dotiert) oder knapp über dem Valenzband ($p$-dotiert) vorstellen.
Dabei erhält man die Dotierung eines Kristalls, indem man fremde Atome in dem
Kristall einbindet.
Wenn man eine positive Dotierung möchte, so fügt man in den Kristall Atome mit einer
niedrigeren Ordnungszahl ein, für die negative Dotierung nutzt man  Atome mit
höherer Ordnungszahl.

Fügt man diese beiden Bereiche aneinander, so entsteht dazwischen eine Bereich,
indem die überschüssigen Elektronen aus dem $p$-dotierten Teil in den
$n$-dotierten Bereich abgesaugt werden. 
Dadurch das die Elektronen und Löcher \fehlt
Hierdurch bildet sich durch die Ionenrümpfe ein elektrisches Feld aus, welches
der weitern Verbreiterung der Verarmungszone entgegen wirkt.
Es entspricht somit einer Diode.
Wenn man diese nun in Sperrichtung betreibt, kann man die Verarmungszone weiter
verbreiter.
Das ist erwünscht, da nur in diesen Bereich des Detektors Ereignisse auch
gemessen werden können.

\chapter{Durchführung}

%TODO

\chapter{Auswertung}

%TODO

\begin{figure}
    \centering
    \includegraphics[width=\plotwidth]{Ge_calib}
    \caption{%
        %
    }
    \label{fig:}
\end{figure}

\begin{figure}
    \centering
    \includegraphics[width=\plotwidth]{Ge_calib-peaks}
    \caption{%
        %
    }
    \label{fig:}
\end{figure}

\begin{tabular}{SSSSSS}
    {Nummer} & {Kanal} & {Breite $\Gamma$} & {Fläche} & {rel.\ Intens.} &
    {$E_\gamma$ / \si{\kilo\electronvolt}} \\
    \midrule
    %< for row in ge_co_fits_table >%
    << ' & '.join(row) >> \\
    %< endfor >%
\end{tabular}


Germanium
\[
    \text{Energie} =
    \text{Kanalnummer} \cdot \SI{<< ge_slope >>}{\kilo\electronvolt}
    +
    \SI{<< ge_offset >>}{\kilo\electronvolt} \,.
\]

\begin{figure}
    \centering
    \includegraphics[width=\plotwidth]{ge_channels}
    \caption{%
        %
    }
    \label{fig:}
\end{figure}

Szintillationszähler
\[
    \text{Energie} =
    \text{Kanalnummer} \cdot \SI{<< sz_slope >>}{\kilo\electronvolt}
    +
    \SI{<< sz_offset >>}{\kilo\electronvolt} \,.
\]

\begin{figure}
    \centering
    \includegraphics[width=\plotwidth]{Sz-CO-peaks}
    \caption{%
        %
    }
    \label{fig:}
\end{figure}

\begin{figure}
    \centering
    \includegraphics[width=\plotwidth]{Sz-CS-peaks}
    \caption{%
        %
    }
    \label{fig:}
\end{figure}

\begin{figure}
    \centering
    \includegraphics[width=\plotwidth]{Sz-EU-peaks}
    \caption{%
        %
    }
    \label{fig:}
\end{figure}

\begin{figure}
    \centering
    \includegraphics[width=\plotwidth]{sz_channels}
    \caption{%
        %
    }
    \label{fig:}
\end{figure}

\[
    \text{Const} = \num{<< ge_const >>}\,\sqrt{\si{\kilo\electronvolt}}
    ,\qquad
    \Delta E_\mathrm e = \SI{<< ge_electronic_width >>}{\kilo\electronvolt}
\]

\begin{figure}
    \centering
    \includegraphics[width=\plotwidth]{halbwertsbreite_ge}
    \caption{%
        %
    }
    \label{fig:}
\end{figure}


\begin{figure}
    \centering
    \includegraphics[width=\plotwidth]{ge_efficiency}
    \caption{%
        %
    }
    \label{fig:}
\end{figure}

\begin{tabular}{SSSS}
    {Energie / \si{\kilo\electronvolt}} & {rel.\ Intens. erw.} & {rel.\ Intens.
mess.} & {Effizienz} \\
    \midrule
    %< for row in ge_efficiency_table >%
    << ' & '.join(row) >> \\
    %< endfor >%
\end{tabular}

\chapter{Ergebnis}

%TODO


%%%%%%%%%%%%%%%%%%%%%%%%%%%%%%%%%%%%%%%%%%%%%%%%%%%%%%%%%%%%%%%%%%%%%%%%%%%%%%%
%                                   Anhang                                    %
%%%%%%%%%%%%%%%%%%%%%%%%%%%%%%%%%%%%%%%%%%%%%%%%%%%%%%%%%%%%%%%%%%%%%%%%%%%%%%%

\begin{appendix}

%TODO

\end{appendix}

\end{document}

% vim: spell spelllang=de tw=79
