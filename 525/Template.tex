\documentclass[11pt, ngerman, fleqn, DIV=15, headinclude, BCOR=2cm]{scrreprt}

\usepackage{../../header}

\usepackage{placeins}
%\usepackage[maxfloats=50]{morefloats}

\usepackage{csquotes}

\usepackage{tikz}
\usetikzlibrary{chains}
\usetikzlibrary{shapes.geometric}

\tikzset{device/.style={
                rectangle,
                minimum size=6mm,
                draw=black
            },
            monitor/.style={
                rectangle,
                rounded corners=2mm,
                minimum size=6mm,
                draw=black
            },
        }

\usepackage{pgfplots}
\pgfplotsset{
    compat=1.9,
    width=0.8\linewidth,
    xticklabel style={/pgf/number format/use comma},
    yticklabel style={/pgf/number format/use comma},
}
\usepgfplotslibrary{polar}

\usepgfplotslibrary{external}
\tikzexternalize[mode=list and make]
\tikzsetexternalprefix{Abbildung-}

\DeclareSIUnit{\skt}{SKT}

\usepackage{booktabs}

\usepackage{subcaption}

\hypersetup{
    pdftitle=
}

\newcommand{\plotwidth}{0.8\linewidth}

\subject{Praktikumsprotokoll}
\title{Nukleare Elektronik und Lebensdauermessung}
\subtitle{Versuch P525 -- Universität Bonn}
\author{
	Frederike Schrödel \\
	\small{\href{mailto:fschroedel@gmx.de}{fschroedel@gmx.de}}
	\and
	Simon Schlepphorst \\
	\small{\href{mailto:s2@uni-bonn.de}{s2@uni-bonn.de}}
}

\date{2015-12-08 bis 12-09}

\publishers{Tutor: Philipp Hoffmeister
}

\begin{document}

\maketitle

\begin{abstract}
	% TODO
\end{abstract}


\tableofcontents

\chapter{Theorie}

% TODO

\section{Szintillationsdetektor}

Mit einem Szintillationsdetektor lassen sich Intensität und Energie von
ionisierender Strahlung messen.
Er besteht aus einem Szintillator und einem Photomultiplier.
Der Szintillator ist ein dotierter Einkristall, der von energiereicher
Strahlung zum Fluoreszieren angeregt werden kann und hinreichend
durchlässig für das von ihm emittierte Licht ist.

Innerhalb des vor Lichteinfall geschützten Szintillators entstehen durch
ionisierende Strahlung Lichtblitze. 
% FIXME Sinn?
Diese entstehen dadurch, dass die Strahlung Atome durch die
Effekt~\ref{sec:WW-strahlung-Materie} anregt, welche entweder direkt
wieder rekombinieren und Photonen aussenden oder über Stöße vorher einen Teil
ihrer Energie abgeben.
Somit ist die Intensität dieser Blitze abhängig von der Energie der einfallenden Strahlung.

\subsection{Photomultiplier}

Aufgrund des Photoeffektes werden an der Photokathode des Photomultipliers
Elektronen ausgelöst.

Durch den Aufbau des Photomultipliers kommt es zu einem Lawineneffekt,
welcher die Elektronen vervielfacht und somit eine Detektion erleichtert.
Die Amplitude des so entstanden Strompulses ist somit auch proportional zu der
Energie der Eingangsstrahlung.

\subsection{Energie und Zeitauflösung}

Um mit dieser Art Detektor möglichst gute Auflösungen zu erhalten, greift man
das Signal -- je nachdem ob man die Energie- oder die Zeitauflösung optimieren
will --
an einer Dynode bzw.\ an der Anode des Photomultipliers ab.
Der Vorteil der Dynode für die Energieauflösung liegt dadrin, dass noch keine
Sättigung herrscht. Hierdurch ist die Proportionalität zwischen Signalhöhe und
Energie nicht beeinträchtigt. Da das Signal allerdings nur sehr langsam
ansteigt, man spricht vom Slow-Signal, ist die Zeitauflösung beeinträchtigt.
Deswegen benutzt man hier das sogenannte Fast-Signal. Es wird an der Anode
abgegriffen, da dort Sättigung vorliegt, wodurch das Signal schnell ansteigt.
Die Amplitude ist allerdings nicht mehr proportional zur Energie.
Um die Zeitauflösung ermitteln zu können nutzt man eine
Slow-Fast-Koinzidenzschaltung. Man misst bekannte Linien und bestimmt aus den
angepassten Funktionen die Halbwertsbreite.

%TODO für Nukleare elektronik
Für die Zeitkalibrirung nutzt man
am besten ein Isotope, welches einen $\beta^+$-Zerfall besitzt. Dadurch kann
man gewährleisten, dass gleichzeitig zwei Photonen mit bekannter Wellenlänge
von \SI{511}{\kilo\electronvolt} durch die Rekombination von Elektron und
Positron entstehen.

\subsection{Splitter}
Im laufe des Versuches wird es nötig sein, dass das Signal aufgeteilt wird,
ohne dabei verformt zu werden. Hierfür nutzt man ein Bauteil, welches Splitter
genannt wird. 

\subsection{Einkanalanalysator}
Der Einkanalanalysator, auch Single Channel Analyser, kurz SCA, genannt ist ein
Bauteil, welches ähnlich eines Diskrimiators eine untere, aber auch eine obere
Grenze besitzt. Wenn ein Signal die untere Grenze übersteigt, aber die obere
Grenze nicht erreicht, so sendet der SCA ein digitales Signal aus, sobald das
Signal wider unter die untere Grenze singt.

\subsection{Vielkanalanalysator}
Um ein nach Amplituden sortiertes Spektrum zu erhalten, nutzen wir ein MCA.
Dies ist ein Bauteil, welches einkommende Signale nach Amplitude sortiert und
dann den Kanal, welcher der Impulshöhe entspricht, hoch setzt. So erhalten wir ein
Histogramm. Der MCA ist aus mhreren SCAs aufgebaut.

\subsection{Contsant Fraction Diskrimiantor}
Einen CFD ist ein Diskrimiantor und somit in der Lage ein analoges
Eingangsignal in ein digitales Ausgangssignal umwandeln, wenn ein Signal einen
bestimmten Wert übersteigt.
Der Constant Fraction Diskriminator ist allerdings auch in der Lage Signale
verschiedener größe zu detektieren, sofern sie die selbe Form haben, da diese
dann die gleiche Anstiegszeit haben. Das digitale Signal wird abgegeben, sobald
das analoge Signal einen bestimmten Bruchteil seiner Amplitude erreicht.

\subsection{Verstärker}
Ein Verstärker hat wie der Name schon sagt die Aufgabe ein Signal zu
Verstärken. Man nutzt meist einen Vor- und einen Hauptverstärker um zu
gewährleisten das die Verstärkung linear ist. 

\subsection{Delays}
Delays sollen die elektrischen Signale verzögern. Da die länge eines Kabels
proportional zu dessen Verzögerung ist, kann man ein Kabel mit entsprechender
Länge nutzten um eine bestimmte Verzögerung zu erhalten.

\subsection{Koinzidenzeinheit}
Mit der Koinzidenzeinheit lässt sich überprüfen, ob zwei Signale zeitnah
nacheinander auftreffen. Hierzu werden die Eingangssignale über ein logisches
AND mit einander verknüpft und im Falle eines Überlapes ein Ausgangssignal
ausgegeben. Dadurch wird aber direkt die Zeitauflösung auf die doppelte
Impulslänge begrenzt.

\subsection{Zeit-Amplituden-Wandler}
Der Zeit-Amplituden-Wandeler (tims-to-amplitude-converter oder auch TAC) soll
aus der der zeitlichen Differenz zweier Signale ein Signal erstellen, welches
in der Amplitude proportional zu der Dauer ist. Das lässt sich bespielsweise da
durch realisieren, dass durch Eintreffen des ersten Signals ein Kondensator
aufgeladen wird. Der Ladevorgang wird beendet sobald das zweite Signal
eintrifft. Bei dem Entladen ist das resultierende Signal ist in der höhe
proportional zu der Zeitdifferenz. 

\chapter{Aufbau und Durchführung}
%TODO

\section{Slow-Koinzidenzkreis einstellen}
%Na-22
%Slow-Kreis

\begin{figure}
	\centering
	\begin{subfigure}{0.49 \textwidth}
		\includegraphics[width=\textwidth]{TEK00007}
		\caption{%
			links
		}
		\label{fig:slow_signal-li}
	\end{subfigure}
	\begin{subfigure}{0.49 \textwidth}
		\includegraphics[width=\textwidth]{TEK00005}
		\caption{%
			rechts
		}
		\label{fig:slow_signal-re}
	\end{subfigure}
	\caption{%
		Signal nach dem Vorverstärker
	}
	\label{fig:slow_signal}
\end{figure}

\begin{figure}
	\centering
	\begin{subfigure}{0.49 \textwidth}
		\includegraphics[width=\textwidth]{TEK00012}
	\end{subfigure}
	\begin{subfigure}{0.49 \textwidth}
		\includegraphics[width=\textwidth]{TEK00013}
	\end{subfigure}
	\caption{%
		Signal nach dem Hauptverstärker
	}
	\label{fig:slow_signal_hv}
\end{figure}

\begin{figure}
	\centering
	\begin{subfigure}{0.49 \textwidth}
		\includegraphics[width=\textwidth]{TEK00014}
		\caption{%
			links
		}
		\label{fig:slow_signal_sca_trig-li}
	\end{subfigure}
	\begin{subfigure}{0.49 \textwidth}
		\includegraphics[width=\textwidth]{TEK00015}
		\caption{%
			rechts
		}
		\label{fig:slow_signal_sca_trig-re}
	\end{subfigure}
	\caption{%
		Signal nach dem Hauptverstärker mit SCA getriggert\\
		GDG ist in beiden Abbildungen gleich eingestellt
	}
	\label{fig:slow_signal_sca_trig}
\end{figure}

\begin{figure}
	\centering
	\begin{subfigure}{0.49 \textwidth}
		\includegraphics[width=\textwidth]{plot_spektrum_li_na_raw}
		\caption{%
			links
		}
		\label{fig:slow_signal_sca_trig-li_plot}
	\end{subfigure}
	\begin{subfigure}{0.49 \textwidth}
		\includegraphics[width=\textwidth]{plot_spektrum_re_na_raw}
		\caption{%
			rechts
		}
		\label{fig:slow_signal_sca_trig-re_plot}
	\end{subfigure}
	\caption{%
		$^{22}\text{Na}$-Spektrum mit grob eingestelltem
		Hauptverstärker
	}
	\label{fig:slow_signal_sca_trig_plot}
\end{figure}

Anschließend haben wir den Hauptverstärker für beide Seiten jeweils so
eingestellt, dass der \SI{511}{\kilo\electronvolt} Peak aus dem
$^{22}\text{Na}$ am rechten Rand des aufgenommen Spektrums liegt und das
$^{133}\text{Ba}$-Spektrum ebenfalls vollständig aufgenommen wird. Dazu haben
wir die Proben mehrmals gewechselt und den Verstärker feinjustiert, bis wir die
in Abbildung~\ref{fig:slow_signal_hv_eingestellt_plot} und
Abbildung~\ref{fig:ba_slow_signal_hv_eingestellt_plot} zu sehenden Spektren
erhalten haben.


\begin{figure}
	\centering
	\begin{subfigure}{0.49 \textwidth}
		\includegraphics[width=\textwidth]{TEK00018}
		\caption{%
			links
		}
		\label{fig:slow_signal_hv_eingestellt-li}
	\end{subfigure}
	\begin{subfigure}{0.49 \textwidth}
		\includegraphics[width=\textwidth]{TEK00017}
		\caption{%
			rechts
		}
		\label{fig:slow_signal_hv_eingestellt-re}
	\end{subfigure}
	\caption{%
		Signal nach dem Hauptverstärker mit SCA getriggert\\
		Nach der Spektrenauswahl durch den Verstärker
	}
	\label{fig:slow_signal_hv_eingestellt}
\end{figure}

\begin{figure}
	\centering
	\begin{subfigure}{0.49 \textwidth}
		\includegraphics[width=\textwidth]{plot_spektrum_li_na}
		\caption{%
			links
		}
		\label{fig:slow_hv_eingestellt-li_plot}
	\end{subfigure}
	\begin{subfigure}{0.49 \textwidth}
		\includegraphics[width=\textwidth]{plot_spektrum_re_na}
		\caption{%
			rechts
		}
		\label{fig:slow_hv_eingestellt-re_plot}
	\end{subfigure}
	\caption{%
		$^{22}\text{Na}$-Spektrum mit fertig eingestelltem
		Hauptverstärker
	}
	\label{fig:slow_signal_hv_eingestellt_plot}
\end{figure}


\begin{figure}
	\centering
	\begin{subfigure}{0.49 \textwidth}
		\includegraphics[width=\textwidth]{TEK00019}
		\caption{%
			links
		}
		\label{fig:slow_signal_sca_eingestellt-li}
	\end{subfigure}
	\begin{subfigure}{0.49 \textwidth}
		\includegraphics[width=\textwidth]{TEK00020}
		\caption{%
			rechts
		}
		\label{fig:slow_signal_sca_eingestellt-re}
	\end{subfigure}
	\caption{%
		Signal nach dem Hauptverstärker, Schwellen des SCA auf
		$^{22}\text{Na}$-Spektrum angepasst.
	}
	\label{fig:slow_signal_sca_eingestellt}
\end{figure}

\begin{figure}
	\centering
	\begin{subfigure}{0.49 \textwidth}
		\includegraphics[width=\textwidth]{plot_spektrum_filter_li_Na}
		\caption{%
			links
		}
		\label{fig:slow_sca_eingestellt-li_plot}
	\end{subfigure}
	\begin{subfigure}{0.49 \textwidth}
		\includegraphics[width=\textwidth]{plot_spektrum_filter_re_Na}
		\caption{%
			rechts
		}
		\label{fig:slow_sca_eingestellt-re_plot}
	\end{subfigure}
	\caption{%
		$^{22}\text{Na}$-Spektrum nach Auswahl durch SCA
	}
	\label{fig:slow_signal_sca_eingestellt_plot}
\end{figure}

Durch das Einstellen des SCA haben wir links die Kanäle
\numrange{<< li_Na_lower_channel >>}{<< li_Na_upper_channel >>} und rechts
die Kanäle
\numrange{<< re_Na_lower_channel >>}{<< re_Na_upper_channel >>} ausgewählt.

\begin{figure}
	\centering
	\begin{subfigure}{0.49 \textwidth}
		\includegraphics[width=\textwidth]{TEK00021}
	\end{subfigure}
	\begin{subfigure}{0.49 \textwidth}
		\includegraphics[width=\textwidth]{TEK00022}
	\end{subfigure}
	\caption{%
		Koinzidenz der SCA gegeneinander. Kanal 1 ist der linke
		Detektor, Kanal 2 der rechte.
	}
	\label{fig:slow_signal_sca_koinzidenz}
\end{figure}


%Fast-Kreis

\begin{figure}
	\centering
	\begin{subfigure}{0.49 \textwidth}
		\includegraphics[width=\textwidth]{TEK00027}
		\caption{%
			links
		}
		\label{fig:fast_signal-li}
	\end{subfigure}
	\begin{subfigure}{0.49 \textwidth}
		\includegraphics[width=\textwidth]{TEK00028}
		\caption{%
			rechts
		}
		\label{fig:fast_signal-re}
	\end{subfigure}
	\caption{%
		Fast Pulse vom Photomultiplier. Auf der rechten Seite haben wir
		den recht kurz aufleuchtenden Puls mit dem langsam speichernden
		Oszilloskop nicht getroffen. Er sah aber genauso aus wie links.
	}
	\label{fig:fast_signal}
\end{figure}

\begin{figure}
	\centering
	\begin{subfigure}{0.49 \textwidth}
		\includegraphics[width=\textwidth]{TEK00031}
		\caption{%
			links
		}
		\label{fig:fast_signal_cfd_trig-li}
	\end{subfigure}
	\begin{subfigure}{0.49 \textwidth}
		\includegraphics[width=\textwidth]{TEK00032}
		\caption{%
			rechts
		}
		\label{fig:fast_signal_cfd_trig-re}
	\end{subfigure}
	\caption{%
		Signal nach dem Hauptverstärker mit CFD getriggert
	}
	\label{fig:fast_signal_cfd_trig}
\end{figure}

\begin{figure}
	\centering
	\begin{subfigure}{0.49 \textwidth}
		\includegraphics[width=\textwidth]{plot_spektrum_cfd_li_Na}
		\caption{%
			links
		}
		\label{fig:fast_signal_cfd_plot-li}
	\end{subfigure}
	\begin{subfigure}{0.49 \textwidth}
		\includegraphics[width=\textwidth]{plot_spektrum_cfd_re_Na}
		\caption{%
			rechts
		}
		\label{fig:fast_signal_cfd_plot-re}
	\end{subfigure}
	\caption{%
		$^{22}\text{Na}$-Spektrum nach dem Setzen der CFD Schwellen
	}
	\label{fig:fast_signal_cfd_plot}
\end{figure}

Durch das Einstellen des CFD haben wir links die Kanäle
\numrange{<< li_Na_lower_cfd_channel >>}{<< li_Na_upper_cfd_channel >>} und rechts
die Kanäle
\numrange{<< re_Na_lower_cfd_channel >>}{<< re_Na_upper_cfd_channel >>} ausgewählt.


\begin{figure}
	\centering
	\begin{subfigure}{0.49 \textwidth}
		\includegraphics[width=\textwidth]{TEK00033}
		\caption{%
			$\Delta t = 0$
		}
		\label{fig:fast_signal_tac_koinzidenz-t0}
	\end{subfigure}
	\begin{subfigure}{0.49 \textwidth}
		\includegraphics[width=\textwidth]{TEK00034}
		\caption{%
			$\Delta t = \SI{70}{\nano\second}$
		}
		\label{fig:fast_signal_tac_koinzidenz-t5-16}
	\end{subfigure}
	\caption{%
		Koinzidenz im TAC
	}
	\label{fig:fast_signal_tac_koinzidenz}
\end{figure}

\begin{figure}
	\centering
	\includegraphics[width=\textwidth]{TEK00035}
	\caption{%
		Slow-Fast Koinzidenz
	}
	\label{fig:slow_fast_koinzidenz}
\end{figure}

\fehlt%TODO Promp Kurven

\clearpage

\section{Messung der Lebensdauer}
%Ba-133
%Slow-Kreis

\begin{figure}
	\centering
	\begin{subfigure}{0.49 \textwidth}
		\includegraphics[width=\textwidth]{plot_spektrum_li_ba}
		\caption{%
			links
		}
		\label{fig:ba_slow_hv_eingestellt-li_plot}
	\end{subfigure}
	\begin{subfigure}{0.49 \textwidth}
		\includegraphics[width=\textwidth]{plot_spektrum_re_ba}
		\caption{%
			rechts
		}
		\label{fig:ba_slow_hv_eingestellt-re_plot}
	\end{subfigure}
	\caption{%
		$^{133}\text{Ba}$-Spektrum mit fertig eingestelltem
		Hauptverstärker
	}
	\label{fig:ba_slow_signal_hv_eingestellt_plot}
\end{figure}

\begin{figure}
	\centering
	\begin{subfigure}{0.49 \textwidth}
		\includegraphics[width=\textwidth]{TEK00037}
		\caption{%
			links, \SI{81}{\kilo\electronvolt}
		}
		\label{fig:ba_slow_signal_sca_eingestellt-li}
	\end{subfigure}
	\begin{subfigure}{0.49 \textwidth}
		\includegraphics[width=\textwidth]{TEK00038}
		\caption{%
			rechts, \SI{356}{\kilo\electronvolt}
		}
		\label{fig:ba_slow_signal_sca_eingestellt-re}
	\end{subfigure}
	\caption{%
		Signal dem Hauptverstärker, Schwellen des SCA auf die Linien
		des $^{133}\text{Ba}$-Spektrums angepasst
	}
	\label{fig:ba_slow_signal_sca_eingestellt}
\end{figure}

\begin{figure}
	\centering
	\begin{subfigure}{0.49 \textwidth}
		\includegraphics[width=\textwidth]{plot_spektrum_filter_li_Ba}
		\caption{%
			links
		}
		\label{fig:ba_slow_sca_eingestellt-li_plot}
	\end{subfigure}
	\begin{subfigure}{0.49 \textwidth}
		\includegraphics[width=\textwidth]{plot_spektrum_filter_re_Ba}
		\caption{%
			rechts
		}
		\label{fig:ba_slow_sca_eingestellt-re_plot}
	\end{subfigure}
	\caption{%
		$^{133}\text{Ba}$-Spektrum nach Auswahl durch SCA
	}
	\label{fig:ba_slow_signal_sca_eingestellt_plot}
\end{figure}

Durch das Einstellen des SCA haben wir links die Kanäle
\numrange{<< li_Ba_lower_channel >>}{<< li_Ba_upper_channel >>} und rechts
die Kanäle
\numrange{<< re_Ba_lower_channel >>}{<< re_Ba_upper_channel >>} ausgewählt.


\begin{figure}
	\centering
	\begin{subfigure}{0.49 \textwidth}
		\includegraphics[width=\textwidth]{TEK00039}
	\end{subfigure}
	\begin{subfigure}{0.49 \textwidth}
		\includegraphics[width=\textwidth]{TEK00040}
	\end{subfigure}
	\caption{%
		Koinzidenz der SCA gegeneinander. Kanal 1 ist der linke
		Detektor, Kanal 2 der rechte.
	}
	\label{fig:ba_slow_signal_sca_koinzidenz}
\end{figure}


%Fast-Kreis

\begin{figure}
	\centering
	\begin{subfigure}{0.49 \textwidth}
		\includegraphics[width=\textwidth]{TEK00043}
		\caption{%
			links
		}
		\label{fig:ba_fast_signal_cfd_trig-li}
	\end{subfigure}
	\begin{subfigure}{0.49 \textwidth}
		\includegraphics[width=\textwidth]{TEK00045}
		\caption{%
			rechts
		}
		\label{fig:ba_fast_signal_cfd_trig-re}
	\end{subfigure}
	\caption{%
		Signal nach dem Hauptverstärker mit CFD getriggert
	}
	\label{fig:ba_fast_signal_cfd_trig}
\end{figure}

\begin{figure}
	\centering
	\begin{subfigure}{0.49 \textwidth}
		\includegraphics[width=\textwidth]{plot_spektrum_cfd_li_Ba}
		\caption{%
			links
		}
		\label{fig:ba_fast_signal_cfd_plot-li}
	\end{subfigure}
	\begin{subfigure}{0.49 \textwidth}
		\includegraphics[width=\textwidth]{plot_spektrum_cfd_re_Ba}
		\caption{%
			rechts
		}
		\label{fig:ba_fast_signal_cfd_plot-re}
	\end{subfigure}
	\caption{%
		$^{133}\text{Ba}$-Spektrum nach dem Setzen der CFD Schwellen
	}
	\label{fig:ba_fast_signal_cfd_plot}
\end{figure}

Durch das Einstellen des CFD haben wir links die Kanäle
\numrange{<< li_Ba_lower_cfd_channel >>}{<< li_Ba_upper_cfd_channel >>} und rechts
die Kanäle
\numrange{<< re_Ba_lower_cfd_channel >>}{<< re_Ba_upper_cfd_channel >>} ausgewählt.


\begin{figure}
	\centering
	\includegraphics[width=\textwidth]{TEK00046}
	\caption{%
		Slow-Fast Koinzidenz
	}
	\label{fig:ba_slow_fast_koinzidenz}
\end{figure}

\fehlt%TODO Lebenszeit Plot

\chapter{Auswertung}
%TODO
\section{Energiekalibrierung}

\subsection{Anpassen der Spektren}
\begin{figure}[h]
	\centering
	\begin{subfigure}{0.49 \textwidth}
		\includegraphics[width=\textwidth]{plot_peaks_li_na}
		\caption{%
			links
		}
		\label{fig:na_peaks-li_plot}
	\end{subfigure}
	\begin{subfigure}{0.49 \textwidth}
		\includegraphics[width=\textwidth]{plot_peaks_re_na}
		\caption{%
			rechts
		}
		\label{fig:na_peaks-re_plot}
	\end{subfigure}
	\caption{%
		$^{22}\text{Na}$-Spektren mit angepassten Gaußkurven
	}
	\label{fig:na_peaks_plot}
\end{figure}
\begin{figure}[h]
	\centering
	\begin{subfigure}{0.49 \textwidth}
		\includegraphics[width=\textwidth]{plot_peaks_li_ba}
		\caption{%
			links
		}
		\label{fig:ba_peaks-li_plot}
	\end{subfigure}
	\begin{subfigure}{0.49 \textwidth}
		\includegraphics[width=\textwidth]{plot_peaks_re_ba}
		\caption{%
			rechts
		}
		\label{fig:ba_peaks-re_plot}
	\end{subfigure}
	\caption{%
		$^{133}\text{Ba}$-Spektren mit angepassten Gaußkurven
	}
	\label{fig:ba_peaks_plot}
\end{figure}

Um die Energieeichung erstellen zu können passen wir zunächst Gaußkurven an die
\SIlist{31;81;356;511}{\kilo\electronvolt} Linien an (vgl.
Abbildung~\ref{fig:na_peaks_plot} und \ref{fig:ba_peaks_plot}).
Die einzelnen Anpassungen sind detailiert, mit dem jeweils als Grundlage
gewählten Datenbereich, auf den Abbildungen in
Anhang~\ref{anhang-energiekalibrierung} zu sehen.
Da nur eine recht kurze Messzeit von etwa \SI{2}{\minute} gewählt wurde sind
besonders die schmalen $^{133}\text{Ba}$-Linien nicht sehr gut ausgeformt, was
eine Gaußanpassung erschwert.

Die Parameter der Anpassungen sind in Tabelle~\ref{tab:energiekalibrierung}
aufgeführt. Mit diesen wurden dann die in
Abbildung~\ref{fig:energiekalibrierung_plot} zu sehenden Energiekalibrierungen
je Detektor erstellt. Die Datenpunkte sind mit Fehler dargestellt, allerdings
ist dieser zu klein um in dieser Auflösung gesehen zu werden.

\begin{figure}[h]
    \begin{minipage}[t]{0.45\textwidth}
	\centering
	\begin{tabular}{SSS}
		{Kanal} &
		{FWHM} &
		{E / \si{\kilo\electronvolt}}\\
		\midrule
		%< for row in li_energy_calibration_table: ->%
		<< ' & '.join(row) >> \\
		%< endfor ->%
	\end{tabular}
    \end{minipage}
    \hfill
    \begin{minipage}[t]{0.45\textwidth}
        \centering
        \begin{tabular}{SSS}
		{Kanal} &
		{FWHM} &
		{E / \si{\kilo\electronvolt}}\\
		\midrule
		%< for row in re_energy_calibration_table: ->%
		<< ' & '.join(row) >> \\
		%< endfor ->%
	\end{tabular}
    \end{minipage}
	\caption{%
		Anpassungsparameter für die Energiekalibrierung
	}
	\label{tab:energiekalibrierung}
\end{figure}

\begin{figure}[h]
	\centering
	\begin{subfigure}{0.49 \textwidth}
		\includegraphics[width=\textwidth]{plot_energy_calibrate_fit_li}
		\caption{%
			links
		}
		\label{fig:energiekalibrierung-li_plot}
	\end{subfigure}
	\begin{subfigure}{0.49 \textwidth}
		\includegraphics[width=\textwidth]{plot_energy_calibrate_fit_re}
		\caption{%
			rechts
		}
		\label{fig:energiekalibrierung-re_plot}
	\end{subfigure}
	\caption{%
		Energiekalibrierung
	}
	\label{fig:energiekalibrierung_plot}
\end{figure}

Als Ergebnis für die Geradenanpassung erhalten wir für den linken und rechten
Detektor:
\begin{align}
	E_l\del x = m x + n
	&&\text{mit } m = \SI{<< li_energy_slope >>}{\kilo\electronvolt}\text{, }
	\quad n = \SI{<< li_energy_offset >>}{\kilo\electronvolt}
\end{align}
\begin{align}
	E_r\del x = m x + n
	&&\text{mit } m = \SI{<< re_energy_slope >>}{\kilo\electronvolt}\text{, }
	\quad n = \SI{<< re_energy_offset >>}{\kilo\electronvolt}
\end{align}

\subsection{Energieauflösung}

Die absolute und relative Energieauflösung der einzelnen Linien
stellen wir in Abbildung~\ref{tab:energieaufloesung} dar.
Dabei fällt auf, dass die Auflösung stark von der Energie abhängt.

Da es in diesem Semester keine parallelen Gruppen an einem anderen Aufbau gab,
können wir die Energieauflösung auch nicht mit diesen vergleichen.

\begin{figure}[h]
    \begin{minipage}[t]{0.45\textwidth}
	\centering
	\begin{tabular}{SSS}
		{E / \si{\kilo\electronvolt}} &
		{FWHM / \si{\kilo\electronvolt}} &
		{rel. Auflösung}\\
		\midrule
		%< for row in li_energy_resolution_table: ->%
		<< ' & '.join(row) >> \\
		%< endfor ->%
	\end{tabular}
    \end{minipage}
    \hfill
    \begin{minipage}[t]{0.45\textwidth}
        \centering
        \begin{tabular}{SSS}
		{E / \si{\kilo\electronvolt}} &
		{FWHM / \si{\kilo\electronvolt}} &
		{rel. Auflösung}\\
		\midrule
		%< for row in re_energy_resolution_table: ->%
		<< ' & '.join(row) >> \\
		%< endfor ->%
	\end{tabular}
    \end{minipage}
	\caption{%
		Energieauflösung
	}
	\label{tab:energieaufloesung}
\end{figure}


\subsection{Umrechnen der Schwellen}

Mit der Energiekalibrierung können wir nun die beim Aufbau gewählten Schwellen
für SCA und CFD in Energien umrechnen. Das Ergebnis ist in den
Tabellen~\ref{tab:sca_schwellen} und \ref{tab:cfd_schwellen} zu sehen.
\begin{table}[h]
    \centering
    \begin{tabular}{llSS}
        Probe & Seite & {untere Schwelle / \si{\kilo\electronvolt}} & {obere
    Schwelle} / \si{\kilo\electronvolt} \\
        \midrule
        Na & links & << li_Na_lower_energy >> & << li_Na_upper_energy >> \\
        Na & rechts & << re_Na_lower_energy >> & << re_Na_upper_energy >> \\
        Ba & links & << li_Ba_lower_energy >> & << li_Ba_upper_energy >> \\
        Ba & rechts & << re_Ba_lower_energy >> & << re_Ba_upper_energy >> \\
    \end{tabular}
    \caption{%
        Lage der SCA Schwellen.
    }
    \label{tab:sca_schwellen}
\end{table}
\begin{table}[h]
    \centering
    \begin{tabular}{llSS}
        Probe & Seite & {untere Schwelle / \si{\kilo\electronvolt}} & {obere
    Schwelle} / \si{\kilo\electronvolt} \\
        \midrule
        Na & links & << li_Na_lower_cfd_energy >> & << li_Na_upper_cfd_energy >> \\
        Na & rechts & << re_Na_lower_cfd_energy >> & << re_Na_upper_cfd_energy >> \\
        Ba & links & << li_Ba_lower_cfd_energy >> & << li_Ba_upper_cfd_energy >> \\
        Ba & rechts & << re_Ba_lower_cfd_energy >> & << re_Ba_upper_cfd_energy >> \\
    \end{tabular}
    \caption{%
        Lage der CFD Schwellen.
    }
    \label{tab:cfd_schwellen}
\end{table}


\clearpage

\section{Zeiteichung}

Für die Zeiteichung haben wir an die mit jeweils \SI{16}{\nano\second} Abstand
aufgenommenen Linien Gaußkurven angepasst. Einen groben Überblick liefert
Abbildung~\ref{fig:zeiteichung_peaks_plot}. Mehr Details, sowie die zur
Grundlage der Anpassungen gewählten Datenbereiche, kann man den Abbildungen in
Anhang~\ref{anhang-zeiteichung} entnehmen.

Mit den in Tabelle~\ref{tab:zeiteichung} und
Abbildung~\ref{fig:zeiteichung_plot} dargestellten Anpassungsparametern können
wir eine Zeiteichung erstellen und erhalten:
\begin{align}
	t\del x = m x + n
	&&\text{mit } m = \SI{<< time_slope >>}{\nano\second}\text{, }
	\quad n = \SI{<< time_offset >>}{\nano\second}
\end{align}
Dabei ist $n$ nur der Vollständigkeit halber aufgeführt. Da wir nur
Zeitdifferenzen messen, können wir den Nullpunkt willkürlich festsetzen.

Indem wir mit der Zeiteichung die Halbwertsbreiten in Zeiten umrechnen und
anschließend über diese mitteln erhalten wir eine Zeitauflösung von
\SI{<< time_resolution >>}{\nano\second}.

\begin{figure}
	\centering
	\includegraphics[width=\textwidth]{plot_time_calibration}
	\caption{%
		Um \SI{16}{\nano\second} versetzte \SI{511}{\kilo\electronvolt}
		Linien aus $^{22}\text{Na}$ zur Zeiteichung
	}
	\label{fig:zeiteichung_peaks_plot}
\end{figure}

\begin{figure}[h]
	\centering
	\begin{tabular}{SSS}
		{Kanal} &
		{FWHM} &
		{Zeit / \si{\nano\second}}\\
		\midrule
		%< for row in time_calibration_table: ->%
		<< ' & '.join(row) >> \\
		%< endfor ->%
	\end{tabular}
	\caption{%
		Anpassungsparameter für die Zeiteichung
	}
	\label{tab:zeiteichung}
\end{figure}

\begin{figure}
	\centering
	\includegraphics[width=\textwidth]{plot_time_calibrate_fit}
	\caption{%
		Zeiteichung
	}
	\label{fig:zeiteichung_plot}
\end{figure}

\section{Bestimmung der Lebenszeit}
\begin{figure}
	\centering
	\includegraphics[width=\textwidth]{plot_langzeit_fit}
	\caption{%
		Lebenszeitkurve
	}
	\label{fig:langzeit_plot}
\end{figure}


\chapter{Ergebnis}
%TODO


%%%%%%%%%%%%%%%%%%%%%%%%%%%%%%%%%%%%%%%%%%%%%%%%%%%%%%%%%%%%%%%%%%%%%%%%%%%%%%%
%                                   Anhang                                    %
%%%%%%%%%%%%%%%%%%%%%%%%%%%%%%%%%%%%%%%%%%%%%%%%%%%%%%%%%%%%%%%%%%%%%%%%%%%%%%%

\begin{appendix}
\chapter{Anhang}

\section{Abbildungen zur Energiekalibrierung} \label{anhang-energiekalibrierung}

\subsection{Linker Detektor}
\begin{figure}[h]
    \centering
    \includegraphics[width=0.6\textwidth]{plot_fit_peak_li_1_na}
    \caption{%
	    \SI{511}{\kilo\electronvolt} Linie im $^{22}\text{Na}$-Spektrum mit
	    Gaußanpassungen
   }
    \label{fig:plot_fit_peak_li_1_Na}
\end{figure}
\begin{figure}[h]
    \centering
    \includegraphics[width=0.6\textwidth]{plot_fit_peak_li_1_ba}
    \caption{%
	    \SI{31}{\kilo\electronvolt} Linie im $^{133}\text{Ba}$-Spektrum mit
	    Gaußanpassungen
   }
    \label{fig:plot_fit_peak_li_1_Ba}
\end{figure}
\begin{figure}[h]
    \centering
    \includegraphics[width=0.7\textwidth]{plot_fit_peak_li_2_ba}
    \caption{%
	    \SI{81}{\kilo\electronvolt} Linie im $^{133}\text{Ba}$-Spektrum mit
	    Gaußanpassungen
   }
    \label{fig:plot_fit_peak_li_2_Ba}
\end{figure}
\begin{figure}[h]
    \centering
    \includegraphics[width=0.7\textwidth]{plot_fit_peak_li_3_ba}
    \caption{%
	    \SI{356}{\kilo\electronvolt} Linie im $^{133}\text{Ba}$-Spektrum mit
	    Gaußanpassungen
   }
    \label{fig:plot_fit_peak_li_3_Ba}
\end{figure}
\clearpage

\subsection{Rechter Detektor}
\begin{figure}[h]
    \centering
    \includegraphics[width=0.7\textwidth]{plot_fit_peak_re_1_na}
    \caption{%
	    \SI{511}{\kilo\electronvolt} Linie im $^{22}\text{Na}$-Spektrum mit
	    Gaußanpassungen
   }
    \label{fig:plot_fit_peak_re_1_Na}
\end{figure}
\begin{figure}[h]
    \centering
    \includegraphics[width=0.7\textwidth]{plot_fit_peak_re_1_ba}
    \caption{%
	    \SI{31}{\kilo\electronvolt} Linie im $^{133}\text{Ba}$-Spektrum mit
	    Gaußanpassungen
   }
    \label{fig:plot_fit_peak_re_1_Ba}
\end{figure}
\begin{figure}[h]
    \centering
    \includegraphics[width=0.7\textwidth]{plot_fit_peak_re_2_ba}
    \caption{%
	    \SI{81}{\kilo\electronvolt} Linie im $^{133}\text{Ba}$-Spektrum mit
	    Gaußanpassungen
   }
    \label{fig:plot_fit_peak_re_2_Ba}
\end{figure}
\begin{figure}[h]
    \centering
    \includegraphics[width=0.7\textwidth]{plot_fit_peak_re_3_ba}
    \caption{%
	    \SI{356}{\kilo\electronvolt} Linie im $^{133}\text{Ba}$-Spektrum mit
	    Gaußanpassungen
   }
    \label{fig:plot_fit_peak_re_3_Ba}
\end{figure}
\clearpage

\section{Abbildungen zur Zeiteichung} \label{anhang-zeiteichung}
\begin{figure}[h]
    \centering
    \includegraphics[width=0.7\textwidth]{plot_fit_peak__1_time}
    \caption{%
	    \SI{511}{\kilo\electronvolt} Linie mit
	    Gaußanpassungen und \SI{16}{\nano\second} Versatz
   }
    \label{fig:plot_fit_peak__1_time}
\end{figure}
\begin{figure}[h]
    \centering
    \includegraphics[width=0.7\textwidth]{plot_fit_peak__2_time}
    \caption{%
	    \SI{511}{\kilo\electronvolt} Linie mit
	    Gaußanpassungen und \SI{32}{\nano\second} Versatz
   }
    \label{fig:plot_fit_peak__2_time}
\end{figure}
\begin{figure}[h]
    \centering
    \includegraphics[width=0.7\textwidth]{plot_fit_peak__3_time}
    \caption{%
	    \SI{511}{\kilo\electronvolt} Linie mit
	    Gaußanpassungen und \SI{48}{\nano\second} Versatz
   }
    \label{fig:plot_fit_peak__3_time}
\end{figure}
\begin{figure}[h]
    \centering
    \includegraphics[width=0.7\textwidth]{plot_fit_peak__4_time}
    \caption{%
	    \SI{511}{\kilo\electronvolt} Linie mit
	    Gaußanpassungen und \SI{64}{\nano\second} Versatz
   }
    \label{fig:plot_fit_peak__4_time}
\end{figure}
\begin{figure}[h]
    \centering
    \includegraphics[width=0.7\textwidth]{plot_fit_peak__5_time}
    \caption{%
	    \SI{511}{\kilo\electronvolt} Linie mit
	    Gaußanpassungen und \SI{80}{\nano\second} Versatz
   }
    \label{fig:plot_fit_peak__5_time}
\end{figure}
\begin{figure}[h]
    \centering
    \includegraphics[width=0.7\textwidth]{plot_fit_peak__6_time}
    \caption{%
	    \SI{511}{\kilo\electronvolt} Linie mit
	    Gaußanpassungen und \SI{96}{\nano\second} Versatz
   }
    \label{fig:plot_fit_peak__6_time}
\end{figure}
\begin{figure}[h]
    \centering
    \includegraphics[width=0.7\textwidth]{plot_fit_peak__7_time}
    \caption{%
	    \SI{511}{\kilo\electronvolt} Linie mit
	    Gaußanpassungen und \SI{112}{\nano\second} Versatz
   }
    \label{fig:plot_fit_peak__7_time}
\end{figure}
\clearpage










%TODO

\end{appendix}

\end{document}

% vim: spell spelllang=de tw=79
