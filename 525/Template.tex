\documentclass[11pt, ngerman, fleqn, DIV=15, headinclude, BCOR=2cm]{scrreprt}

\usepackage{../../header}

\usepackage{placeins}
%\usepackage[maxfloats=50]{morefloats}

\usepackage{csquotes}

\usepackage{tikz}
\usetikzlibrary{chains}
\usetikzlibrary{shapes.geometric}

\tikzset{device/.style={
                rectangle,
                minimum size=6mm,
                draw=black
            },
            monitor/.style={
                rectangle,
                rounded corners=2mm,
                minimum size=6mm,
                draw=black
            },
        }

\usepackage{pgfplots}
\pgfplotsset{
    compat=1.9,
    width=0.8\linewidth,
    xticklabel style={/pgf/number format/use comma},
    yticklabel style={/pgf/number format/use comma},
}
\usepgfplotslibrary{polar}

\usepgfplotslibrary{external}
\tikzexternalize[mode=list and make]
\tikzsetexternalprefix{Abbildung-}

\DeclareSIUnit{\skt}{SKT}

\usepackage{booktabs}

\usepackage{subcaption}

\hypersetup{
    pdftitle=
}

\newcommand{\plotwidth}{0.8\linewidth}

\subject{Praktikumsprotokoll}
\title{Nukleare Elektronik und Lebensdauermessung}
\subtitle{Versuch P525 -- Universität Bonn}
\author{
	Frederike Schrödel \\
	\small{\href{mailto:fschroedel@gmx.de}{fschroedel@gmx.de}}
	\and
	Simon Schlepphorst \\
	\small{\href{mailto:s2@uni-bonn.de}{s2@uni-bonn.de}}
}

\date{2015-12-08 bis 12-09}

\publishers{Tutor: Philipp Hoffmeister
}

\begin{document}

\maketitle

\begin{abstract}
	%TODO
\end{abstract}


\tableofcontents

\chapter{Theorie}
%TODO


\chapter{Aufbau und Durchführung}
%TODO

\begin{figure}
	\centering
	\begin{subfigure}{0.49 \textwidth}
		\includegraphics[width=\textwidth]{TEK00007}
		\caption{%
			links
		}
		\label{fig:slow_signal-li}
	\end{subfigure}
	\begin{subfigure}{0.49 \textwidth}
		\includegraphics[width=\textwidth]{TEK00005}
		\caption{%
			rechts
		}
		\label{fig:slow_signal-re}
	\end{subfigure}
	\caption{%
		Signal nach dem Vorverstärker
	}
	\label{fig:slow_signal}
\end{figure}

\begin{figure}
	\centering
	\begin{subfigure}{0.49 \textwidth}
		\includegraphics[width=\textwidth]{TEK00012}
	\end{subfigure}
	\begin{subfigure}{0.49 \textwidth}
		\includegraphics[width=\textwidth]{TEK00013}
	\end{subfigure}
	\caption{%
		Signal nach dem Hauptverstärker
	}
	\label{fig:slow_signal_hv}
\end{figure}

\begin{figure}
	\centering
	\begin{subfigure}{0.49 \textwidth}
		\includegraphics[width=\textwidth]{TEK00014}
		\caption{%
			links
		}
		\label{fig:slow_signal_sca_trig-li}
	\end{subfigure}
	\begin{subfigure}{0.49 \textwidth}
		\includegraphics[width=\textwidth]{TEK00015}
		\caption{%
			rechts
		}
		\label{fig:slow_signal_sca_trig-re}
	\end{subfigure}
	\caption{%
		Signal nach dem Hauptverstärker mit SCA getriggert\\
		GDG ist in beiden Abbildungen gleich eingestellt
	}
	\label{fig:slow_signal_sca_trig}
\end{figure}


\begin{figure}
	\centering
	\begin{subfigure}{0.49 \textwidth}
		\includegraphics[width=\textwidth]{TEK00018}
		\caption{%
			links
		}
		\label{fig:slow_signal_hv_eingestellt-li}
	\end{subfigure}
	\begin{subfigure}{0.49 \textwidth}
		\includegraphics[width=\textwidth]{TEK00017}
		\caption{%
			rechts
		}
		\label{fig:slow_signal_hv_eingestellt-re}
	\end{subfigure}
	\caption{%
		Signal nach dem Hauptverstärker mit SCA getriggert\\
		Nach der Spektrenauswahl durch den Verstärker
	}
	\label{fig:slow_signal_hv_eingestellt}
\end{figure}


\begin{figure}
	\centering
	\begin{subfigure}{0.49 \textwidth}
		\includegraphics[width=\textwidth]{TEK00019}
		\caption{%
			links
		}
		\label{fig:slow_signal_sca_eingestellt-li}
	\end{subfigure}
	\begin{subfigure}{0.49 \textwidth}
		\includegraphics[width=\textwidth]{TEK00020}
		\caption{%
			rechts
		}
		\label{fig:slow_signal_sca_eingestellt-re}
	\end{subfigure}
	\caption{%
		Signal nach dem Hauptverstärker, Schwellen des SCA auf
		$^{22}\text{Na}$-Spektrum angepasst.
	}
	\label{fig:slow_signal_sca_eingestellt}
\end{figure}

\begin{figure}
	\centering
	\begin{subfigure}{0.49 \textwidth}
		\includegraphics[width=\textwidth]{TEK00021}
	\end{subfigure}
	\begin{subfigure}{0.49 \textwidth}
		\includegraphics[width=\textwidth]{TEK00022}
	\end{subfigure}
	\caption{%
		Koinzidenz der SCA gegeneinander. Kanal 1 ist der linke
		Detektor, Kanal 2 der rechte.
	}
	\label{fig:slow_signal_sca_eingestellt}
\end{figure}


\chapter{Auswertung}
%TODO


\chapter{Ergebnis}
%TODO


%%%%%%%%%%%%%%%%%%%%%%%%%%%%%%%%%%%%%%%%%%%%%%%%%%%%%%%%%%%%%%%%%%%%%%%%%%%%%%%
%                                   Anhang                                    %
%%%%%%%%%%%%%%%%%%%%%%%%%%%%%%%%%%%%%%%%%%%%%%%%%%%%%%%%%%%%%%%%%%%%%%%%%%%%%%%

\begin{appendix}

%TODO

\end{appendix}

\end{document}

% vim: spell spelllang=de tw=79
